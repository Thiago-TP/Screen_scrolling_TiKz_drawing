\usepackage{tikz}
\usepackage{calculator, xcolor}

\def\tileLen		{16}
\def\tilemapLen		{30*\tileLen}
\def\tilemapHeight	{15*\tileLen}
\def\bmpLen			{20*\tileLen}

\def\angle{10}			% Ângulo de inclinação com a horizontal
\def\compl{180-\angle}	% Ângulo complementar ao de inclinação
\def\scale{1/\tileLen}	% Correção de escala do desenho

\def\arrayBrick{	% Conjunto de bytes de cor da sprite no tilemap
	20,  20,  20,  20,  20,  20,  20,  0,   20,  20,  20,  20,  20,  20,  20,  0,   
	20,  20,  20,  20,  20,  20,  20,  0,   20,  20,  20,  20,  20,  20,  20,  0,   
	20,  20,  20,  20,  20,  20,  20,  0,   20,  20,  20,  20,  20,  20,  20,  0,   
	0,   0,   0,   0,   0,   0,   0,   0,   0,   0,   0,   0,   0,   0,   0,   0,   
	20,  20,  20,  0,   20,  20,  20,  20,  20,  20,  20,  0,   20,  20,  20,  20,  
	20,  20,  20,  0,   20,  20,  20,  20,  20,  20,  20,  0,   20,  20,  20,  20,  
	20,  20,  20,  0,   20,  20,  20,  20,  20,  20,  20,  0,   20,  20,  20,  20,  
	0,   0,   0,   0,   0,   0,   0,   0,   0,   0,   0,   0,   0,   0,   0,   0,   
	20,  20,  20,  20,  20,  20,  20,  0,   20,  20,  20,  20,  20,  20,  20,  0,   
	20,  20,  20,  20,  20,  20,  20,  0,   20,  20,  20,  20,  20,  20,  20,  0,   
	20,  20,  20,  20,  20,  20,  20,  0,   20,  20,  20,  20,  20,  20,  20,  0,   
	0,   0,   0,   0,   0,   0,   0,   0,   0,   0,   0,   0,   0,   0,   0,   0,   
	20,  20,  20,  0,   20,  20,  20,  20,  20,  20,  20,  0,   20,  20,  20,  20,  
	20,  20,  20,  0,   20,  20,  20,  20,  20,  20,  20,  0,   20,  20,  20,  20,  
	20,  20,  20,  0,   20,  20,  20,  20,  20,  20,  20,  0,   20,  20,  20,  20,  
	0,   0,   0,   0,   0,   0,   0,   0,   0,   0,   0,   0,   0,   0,   0,   0   
}

\def\arrayEnemiesOne{	% Conjunto de bytes de cor da sprite no tilemap
	228, 228, 228, 228, 228, 228, 20,  20,  20,  20,  228, 228, 228, 228, 228, 228, 228, 228, 228, 228, 228, 228, 20,  20,  20,  20,  228, 228, 228, 228, 228, 228, 228, 228, 228, 228, 228, 228, 228, 228, 228, 228, 228, 228, 228, 228, 228, 228, 228, 228, 228, 228, 228, 228, 152, 152, 152, 152, 228, 228, 228, 228, 228, 228, 228, 228, 228, 228, 228, 228, 152, 152, 152, 152, 228, 228, 228, 228, 228, 228, 228, 228, 228, 228, 228, 228, 228, 228, 228, 228, 228, 228, 228, 228, 228, 228, 228, 228, 228, 228, 228, 228, 173, 173, 173, 173, 228, 228, 228, 228, 228, 228, 228, 228, 228, 228, 228, 228, 173, 173, 173, 173, 228, 228, 228, 228, 228, 228, 228, 228, 228, 228, 228, 228, 228, 228, 228, 228, 228, 228, 228, 228, 228, 228, 228, 228, 228, 228, 228, 228, 228, 228, 228, 228, 228, 228, 228, 228, 228, 228, 
	228, 228, 228, 228, 228, 20,  20,  20,  20,  20,  20,  228, 228, 228, 228, 228, 228, 228, 228, 228, 228, 20,  20,  20,  20,  20,  20,  228, 228, 228, 228, 228, 228, 228, 228, 228, 228, 228, 228, 228, 228, 228, 228, 228, 228, 228, 228, 228, 228, 228, 228, 228, 228, 152, 152, 152, 152, 152, 152, 228, 228, 228, 228, 228, 228, 228, 228, 228, 228, 152, 152, 152, 152, 152, 152, 228, 228, 228, 228, 228, 228, 228, 228, 228, 228, 228, 228, 228, 228, 228, 228, 228, 228, 228, 228, 228, 228, 228, 228, 228, 228, 173, 173, 173, 173, 173, 173, 228, 228, 228, 228, 228, 228, 228, 228, 228, 228, 173, 173, 173, 173, 173, 173, 228, 228, 228, 228, 228, 228, 228, 228, 228, 228, 228, 228, 228, 228, 228, 228, 228, 228, 228, 228, 228, 228, 228, 228, 228, 228, 228, 228, 228, 255, 228, 228, 228, 228, 228, 228, 228, 
	228, 228, 228, 228, 20,  20,  20,  20,  20,  20,  20,  20,  228, 228, 228, 228, 228, 228, 228, 228, 20,  20,  20,  20,  20,  20,  20,  20,  228, 228, 228, 228, 228, 228, 228, 228, 228, 228, 228, 228, 228, 228, 228, 228, 228, 228, 228, 228, 228, 228, 228, 228, 152, 152, 152, 152, 152, 152, 152, 152, 228, 228, 228, 228, 228, 228, 228, 228, 152, 152, 152, 152, 152, 152, 152, 152, 228, 228, 228, 228, 228, 228, 228, 228, 228, 228, 228, 228, 228, 228, 228, 228, 228, 228, 228, 228, 228, 228, 228, 228, 173, 173, 173, 173, 173, 173, 173, 173, 228, 228, 228, 228, 228, 228, 228, 228, 173, 173, 173, 173, 173, 173, 173, 173, 228, 228, 228, 228, 228, 228, 228, 228, 228, 228, 228, 228, 228, 228, 228, 228, 228, 228, 228, 228, 228, 228, 228, 228, 228, 228, 228, 228, 255, 228, 228, 228, 228, 228, 228, 228, 
	228, 228, 228, 20,  20,  20,  20,  20,  20,  20,  20,  20,  20,  228, 228, 228, 228, 228, 228, 20,  20,  20,  20,  20,  20,  20,  20,  20,  20,  228, 228, 228, 228, 228, 228, 228, 228, 228, 228, 228, 228, 228, 228, 228, 228, 228, 228, 228, 228, 228, 228, 152, 152, 152, 152, 152, 152, 152, 152, 152, 152, 228, 228, 228, 228, 228, 228, 152, 152, 152, 152, 152, 152, 152, 152, 152, 152, 228, 228, 228, 228, 228, 228, 228, 228, 228, 228, 228, 228, 228, 228, 228, 228, 228, 228, 228, 228, 228, 228, 173, 173, 173, 173, 173, 173, 173, 173, 173, 173, 228, 228, 228, 228, 228, 228, 173, 173, 173, 173, 173, 173, 173, 173, 173, 173, 228, 228, 228, 228, 228, 228, 228, 228, 228, 228, 228, 228, 228, 228, 228, 228, 228, 228, 228, 228, 228, 228, 228, 228, 228, 228, 255, 38,  38,  228, 228, 228, 228, 228, 228, 
	228, 228, 20,  0,   0,   20,  20,  20,  20,  20,  20,  0,   0,   20,  228, 228, 228, 228, 20,  0,   0,   20,  20,  20,  20,  20,  20,  0,   0,   20,  228, 228, 228, 228, 228, 228, 228, 228, 228, 228, 228, 228, 228, 228, 228, 228, 228, 228, 228, 228, 152, 80,  80,  152, 152, 152, 152, 152, 152, 80,  80,  152, 228, 228, 228, 228, 152, 80,  80,  152, 152, 152, 152, 152, 152, 80,  80,  152, 228, 228, 228, 228, 228, 228, 228, 228, 228, 228, 228, 228, 228, 228, 228, 228, 228, 228, 228, 228, 173, 0,   0,   173, 173, 173, 173, 173, 173, 0,   0,   173, 228, 228, 228, 228, 173, 0,   0,   173, 173, 173, 173, 173, 173, 0,   0,   173, 228, 228, 228, 228, 228, 228, 228, 228, 228, 228, 228, 228, 228, 228, 228, 228, 228, 228, 228, 228, 255, 228, 228, 228, 228, 255, 38,  38,  228, 228, 228, 228, 255, 228, 
	228, 20,  20,  20,  183, 0,   20,  20,  20,  20,  0,   183, 20,  20,  20,  228, 228, 20,  20,  20,  183, 0,   20,  20,  20,  20,  0,   183, 20,  20,  20,  228, 228, 228, 228, 228, 228, 228, 228, 228, 228, 228, 228, 228, 228, 228, 228, 228, 228, 152, 152, 152, 253, 80,  152, 152, 152, 152, 80,  253, 152, 152, 152, 228, 228, 152, 152, 152, 253, 80,  152, 152, 152, 152, 80,  253, 152, 152, 152, 228, 228, 228, 228, 228, 228, 228, 228, 228, 228, 228, 228, 228, 228, 228, 228, 228, 228, 173, 173, 173, 255, 0,   173, 173, 173, 173, 0,   255, 173, 173, 173, 228, 228, 173, 173, 173, 255, 0,   173, 173, 173, 173, 0,   255, 173, 173, 173, 228, 228, 228, 228, 228, 228, 228, 228, 228, 228, 228, 228, 228, 228, 228, 228, 228, 228, 228, 255, 38,  228, 228, 255, 38,  38,  38,  38,  228, 228, 255, 38,  228, 
	228, 20,  20,  20,  183, 0,   0,   0,   0,   0,   0,   183, 20,  20,  20,  228, 228, 20,  20,  20,  183, 0,   0,   0,   0,   0,   0,   183, 20,  20,  20,  228, 228, 228, 228, 228, 228, 228, 228, 228, 228, 228, 228, 228, 228, 228, 228, 228, 228, 152, 152, 152, 253, 80,  80,  80,  80,  80,  80,  253, 152, 152, 152, 228, 228, 152, 152, 152, 253, 80,  80,  80,  80,  80,  80,  253, 152, 152, 152, 228, 228, 228, 228, 228, 228, 228, 228, 228, 228, 228, 228, 228, 228, 228, 228, 228, 228, 173, 173, 173, 255, 0,   0,   0,   0,   0,   0,   255, 173, 173, 173, 228, 228, 173, 173, 173, 255, 0,   0,   0,   0,   0,   0,   255, 173, 173, 173, 228, 228, 228, 228, 228, 228, 228, 228, 228, 228, 228, 228, 228, 228, 228, 228, 228, 228, 228, 255, 38,  38,  228, 255, 38,  38,  38,  38,  228, 255, 38,  38,  228, 
	20,  20,  20,  20,  183, 0,   183, 20,  20,  183, 0,   183, 20,  20,  20,  20,  20,  20,  20,  20,  183, 0,   183, 20,  20,  183, 0,   183, 20,  20,  20,  20,  228, 228, 228, 228, 228, 228, 228, 228, 228, 228, 228, 228, 228, 228, 228, 228, 152, 152, 152, 152, 253, 80,  253, 152, 152, 253, 80,  253, 152, 152, 152, 152, 152, 152, 152, 152, 253, 80,  253, 152, 152, 253, 80,  253, 152, 152, 152, 152, 228, 228, 228, 228, 228, 228, 228, 228, 228, 228, 228, 228, 228, 228, 228, 228, 173, 173, 173, 173, 255, 0,   255, 173, 173, 255, 0,   255, 173, 173, 173, 173, 173, 173, 173, 173, 255, 0,   255, 173, 173, 255, 0,   255, 173, 173, 173, 173, 228, 228, 228, 228, 228, 228, 228, 228, 228, 228, 228, 228, 228, 228, 228, 228, 228, 228, 255, 38,  38,  38,  13,  38,  38,  38,  13,  255, 38,  38,  38,  228, 
	20,  20,  20,  20,  183, 183, 183, 20,  20,  183, 183, 183, 20,  20,  20,  20,  20,  20,  20,  20,  183, 183, 183, 20,  20,  183, 183, 183, 20,  20,  20,  20,  228, 228, 228, 228, 228, 228, 20,  20,  20,  20,  228, 228, 228, 228, 228, 228, 152, 152, 152, 152, 253, 253, 253, 152, 152, 253, 253, 253, 152, 152, 152, 152, 152, 152, 152, 152, 253, 253, 253, 152, 152, 253, 253, 253, 152, 152, 152, 152, 228, 228, 228, 228, 228, 228, 152, 152, 152, 152, 228, 228, 228, 228, 228, 228, 173, 173, 173, 173, 255, 255, 255, 173, 173, 255, 255, 255, 173, 173, 173, 173, 173, 173, 173, 173, 255, 255, 255, 173, 173, 255, 255, 255, 173, 173, 173, 173, 228, 228, 228, 228, 228, 228, 173, 173, 173, 173, 228, 228, 228, 228, 228, 228, 228, 228, 255, 255, 38,  38,  13,  13,  13,  13,  13,  255, 38,  38,  38,  228, 
	20,  20,  20,  20,  20,  20,  20,  20,  20,  20,  20,  20,  20,  20,  20,  20,  20,  20,  20,  20,  20,  20,  20,  20,  20,  20,  20,  20,  20,  20,  20,  20,  228, 228, 228, 20,  20,  20,  20,  20,  20,  20,  20,  20,  20,  228, 228, 228, 152, 152, 152, 152, 152, 152, 152, 152, 152, 152, 152, 152, 152, 152, 152, 152, 152, 152, 152, 152, 152, 152, 152, 152, 152, 152, 152, 152, 152, 152, 152, 152, 228, 228, 228, 152, 152, 152, 152, 152, 152, 152, 152, 152, 152, 228, 228, 228, 173, 173, 173, 173, 173, 173, 173, 173, 173, 173, 173, 173, 173, 173, 173, 173, 173, 173, 173, 173, 173, 173, 173, 173, 173, 173, 173, 173, 173, 173, 173, 173, 228, 228, 228, 173, 173, 173, 173, 173, 173, 173, 173, 173, 173, 228, 228, 228, 228, 13,  13,  38,  38,  13,  13,  255, 255, 38,  13,  13,  38,  38,  13,  228, 
	228, 20,  20,  20,  20,  183, 183, 183, 183, 183, 183, 20,  20,  20,  20,  228, 228, 20,  20,  20,  20,  183, 183, 183, 183, 183, 183, 20,  20,  20,  20,  228, 228, 20,  20,  0,   0,   0,   20,  20,  20,  20,  0,   0,   0,   20,  20,  228, 228, 152, 152, 152, 152, 253, 253, 253, 253, 253, 253, 152, 152, 152, 152, 228, 228, 152, 152, 152, 152, 253, 253, 253, 253, 253, 253, 152, 152, 152, 152, 228, 228, 152, 152, 80,  80,  80,  152, 152, 152, 152, 80,  80,  80,  152, 152, 228, 228, 173, 173, 173, 173, 255, 255, 255, 255, 255, 255, 173, 173, 173, 173, 228, 228, 173, 173, 173, 173, 255, 255, 255, 255, 255, 255, 173, 173, 173, 173, 228, 228, 173, 173, 0,   0,   0,   173, 173, 173, 173, 0,   0,   0,   173, 173, 228, 255, 255, 255, 13,  13,  13,  255, 255, 38,  38,  38,  13,  13,  13,  13,  228, 
	228, 228, 228, 228, 183, 183, 183, 183, 183, 183, 183, 183, 228, 228, 228, 228, 228, 228, 228, 228, 183, 183, 183, 183, 183, 183, 183, 183, 228, 228, 228, 228, 20,  20,  183, 183, 183, 183, 0,   0,   0,   0,   183, 183, 183, 183, 20,  20,  228, 228, 228, 228, 253, 253, 253, 253, 253, 253, 253, 253, 228, 228, 228, 228, 228, 228, 228, 228, 253, 253, 253, 253, 253, 253, 253, 253, 228, 228, 228, 228, 152, 152, 253, 253, 253, 253, 80,  80,  80,  80,  253, 253, 253, 253, 152, 152, 228, 228, 228, 228, 255, 255, 255, 255, 255, 255, 255, 255, 228, 228, 228, 228, 228, 228, 228, 228, 255, 255, 255, 255, 255, 255, 255, 255, 228, 228, 228, 228, 173, 173, 255, 255, 255, 255, 0,   0,   0,   0,   255, 255, 255, 255, 173, 173, 228, 13,  13,  255, 13,  13,  38,  38,  38,  38,  38,  13,  13,  13,  13,  13,  
	228, 228, 228, 228, 183, 183, 183, 183, 183, 183, 183, 183, 0,   0,   228, 228, 228, 228, 0,   0,   183, 183, 183, 183, 183, 183, 183, 183, 228, 228, 228, 228, 20,  20,  20,  20,  20,  20,  20,  20,  20,  20,  20,  20,  20,  20,  20,  20,  228, 228, 228, 228, 253, 253, 253, 253, 253, 253, 253, 253, 80,  80,  228, 228, 228, 228, 80,  80,  253, 253, 253, 253, 253, 253, 253, 253, 228, 228, 228, 228, 152, 152, 152, 152, 152, 152, 152, 152, 152, 152, 152, 152, 152, 152, 152, 152, 228, 228, 228, 228, 255, 255, 255, 255, 255, 255, 255, 255, 0,   0,   228, 228, 228, 228, 0,   0,   255, 255, 255, 255, 255, 255, 255, 255, 228, 228, 228, 228, 173, 173, 173, 173, 173, 173, 173, 173, 173, 173, 173, 173, 173, 173, 173, 173, 13,  255, 13,  13,  255, 13,  13,  38,  38,  38,  13,  13,  13,  255, 255, 255, 
	228, 228, 228, 0,   0,   183, 183, 183, 183, 183, 0,   0,   0,   0,   0,   228, 228, 0,   0,   0,   0,   0,   183, 183, 183, 183, 183, 0,   0,   228, 228, 228, 228, 228, 228, 183, 183, 183, 183, 183, 183, 183, 183, 183, 183, 228, 228, 228, 228, 228, 228, 80,  80,  253, 253, 253, 253, 253, 80,  80,  80,  80,  80,  228, 228, 80,  80,  80,  80,  80,  253, 253, 253, 253, 253, 80,  80,  228, 228, 228, 228, 228, 228, 253, 253, 253, 253, 253, 253, 253, 253, 253, 253, 228, 228, 228, 228, 228, 228, 0,   0,   255, 255, 255, 255, 255, 0,   0,   0,   0,   0,   228, 228, 0,   0,   0,   0,   0,   255, 255, 255, 255, 255, 0,   0,   228, 228, 228, 228, 228, 228, 255, 255, 255, 255, 255, 255, 255, 255, 255, 255, 228, 228, 228, 13,  13,  13,  13,  38,  255, 13,  13,  13,  13,  13,  255, 255, 255, 228, 228, 
	228, 228, 228, 0,   0,   0,   183, 183, 183, 0,   0,   0,   0,   0,   0,   228, 228, 0,   0,   0,   0,   0,   0,   183, 183, 183, 0,   0,   0,   228, 228, 228, 228, 228, 228, 228, 183, 183, 183, 183, 183, 183, 183, 183, 228, 228, 228, 228, 228, 228, 228, 80,  80,  80,  253, 253, 253, 80,  80,  80,  80,  80,  80,  228, 228, 80,  80,  80,  80,  80,  80,  253, 253, 253, 80,  80,  80,  228, 228, 228, 228, 228, 228, 228, 253, 253, 253, 253, 253, 253, 253, 253, 228, 228, 228, 228, 228, 228, 228, 0,   0,   0,   255, 255, 255, 0,   0,   0,   0,   0,   0,   228, 228, 0,   0,   0,   0,   0,   0,   255, 255, 255, 0,   0,   0,   228, 228, 228, 228, 228, 228, 228, 255, 255, 255, 255, 255, 255, 255, 255, 228, 228, 228, 228, 228, 228, 228, 38,  38,  38,  255, 255, 255, 255, 255, 255, 38,  38,  38,  228, 
	228, 228, 228, 228, 0,   0,   0,   228, 228, 0,   0,   0,   0,   0,   228, 228, 228, 228, 0,   0,   0,   0,   0,   228, 228, 0,   0,   0,   228, 228, 228, 228, 228, 0,   0,   0,   0,   0,   228, 228, 228, 228, 0,   0,   0,   0,   0,   228, 228, 228, 228, 228, 80,  80,  80,  228, 228, 80,  80,  80,  80,  80,  228, 228, 228, 228, 80,  80,  80,  80,  80,  228, 228, 80,  80,  80,  228, 228, 228, 228, 228, 80,  80,  80,  80,  80,  228, 228, 228, 228, 80,  80,  80,  80,  80,  228, 228, 228, 228, 228, 0,   0,   0,   228, 228, 0,   0,   0,   0,   0,   228, 228, 228, 228, 0,   0,   0,   0,   0,   228, 228, 0,   0,   0,   228, 228, 228, 228, 228, 0,   0,   0,   0,   0,   228, 228, 228, 228, 0,   0,   0,   0,   0,   228, 228, 228, 38,  38,  38,  38,  228, 228, 228, 228, 228, 228, 38,  38,  38,  38   
}

\def\arrayEnemiesTwo{
	228, 228, 228, 228, 228, 228, 228, 228, 228, 228, 228, 228, 228, 228, 228, 228, 228, 228, 228, 228, 228, 228, 228, 0,   0,   0,   228, 228, 228, 228, 228, 228, 228, 228, 228, 228, 228, 228, 228, 228, 228, 228, 228, 228, 228, 228, 228, 228, 228, 228, 228, 228, 228, 228, 228, 228, 228, 228, 228, 228, 228, 228, 228, 228, 228, 228, 228, 228, 228, 228, 228, 80,  80,  80,  228, 228, 228, 228, 228, 228, 228, 228, 228, 228, 228, 228, 228, 228, 228, 228, 228, 228, 228, 228, 228, 228, 228, 228, 228, 228, 228, 228, 228, 228, 228, 228, 228, 228, 228, 228, 228, 228, 228, 228, 228, 228, 228, 228, 228, 0,   0,   0,   228, 228, 228, 228, 228, 228, 228, 228, 228, 228, 228, 228, 228, 228, 228, 228, 228, 228, 228, 228, 228, 228, 228, 228, 228, 228, 228, 228, 228, 38,  38,  228, 228, 228, 228, 228, 228, 228, 
	228, 228, 228, 228, 228, 228, 0,   0,   0,   0,   228, 228, 228, 228, 228, 228, 228, 228, 228, 228, 0,   0,   0,   0,   0,   0,   0,   0,   228, 228, 228, 228, 228, 228, 228, 228, 228, 228, 0,   0,   0,   0,   228, 228, 228, 228, 228, 228, 228, 228, 228, 228, 228, 228, 80,  80,  80,  80,  228, 228, 228, 228, 228, 228, 228, 228, 228, 228, 80,  80,  80,  80,  80,  80,  80,  80,  228, 228, 228, 228, 228, 228, 228, 228, 228, 228, 80,  80,  80,  80,  228, 228, 228, 228, 228, 228, 228, 228, 228, 228, 228, 228, 0,   0,   0,   0,   228, 228, 228, 228, 228, 228, 228, 228, 228, 228, 0,   0,   0,   0,   0,   0,   0,   0,   228, 228, 228, 228, 228, 228, 228, 228, 228, 228, 0,   0,   0,   0,   228, 228, 228, 228, 228, 228, 228, 228, 228, 228, 38,  38,  13,  38,  38,  13,  38,  38,  228, 228, 228, 228, 
	228, 228, 228, 228, 0,   0,   0,   0,   0,   0,   0,   0,   228, 228, 228, 228, 228, 228, 228, 0,   0,   0,   0,   0,   0,   0,   0,   0,   0,   228, 228, 228, 228, 228, 228, 228, 0,   0,   0,   0,   0,   0,   0,   0,   228, 228, 228, 228, 228, 228, 228, 228, 80,  80,  80,  80,  80,  80,  80,  80,  228, 228, 228, 228, 228, 228, 228, 80,  80,  80,  80,  80,  80,  80,  80,  80,  80,  228, 228, 228, 228, 228, 228, 228, 80,  80,  80,  80,  80,  80,  80,  80,  228, 228, 228, 228, 228, 228, 228, 228, 0,   0,   0,   0,   0,   0,   0,   0,   228, 228, 228, 228, 228, 228, 228, 0,   0,   0,   0,   0,   0,   0,   0,   0,   0,   228, 228, 228, 228, 228, 228, 228, 0,   0,   0,   0,   0,   0,   0,   0,   228, 228, 228, 228, 228, 228, 228, 228, 38,  13,  13,  13,  13,  13,  13,  38,  228, 228, 228, 228, 
	228, 228, 228, 0,   0,   0,   0,   0,   0,   0,   0,   0,   0,   228, 228, 228, 228, 228, 0,   0,   0,   0,   0,   0,   0,   20,  20,  0,   0,   0,   228, 228, 228, 228, 228, 0,   0,   0,   0,   20,  20,  0,   0,   0,   0,   228, 228, 228, 228, 228, 228, 80,  80,  80,  80,  80,  80,  80,  80,  80,  80,  228, 228, 228, 228, 228, 80,  80,  80,  80,  80,  80,  80,  152, 152, 80,  80,  80,  228, 228, 228, 228, 228, 80,  80,  80,  80,  152, 152, 80,  80,  80,  80,  228, 228, 228, 228, 228, 228, 0,   0,   0,   0,   0,   0,   0,   0,   0,   0,   228, 228, 228, 228, 228, 0,   0,   0,   0,   0,   0,   0,   173, 173, 0,   0,   0,   228, 228, 228, 228, 228, 0,   0,   0,   0,   173, 173, 0,   0,   0,   0,   228, 228, 228, 228, 228, 228, 13,  13,  13,  13,  13,  13,  13,  13,  13,  13,  228, 228, 228, 
	228, 228, 0,   0,   0,   0,   0,   0,   0,   20,  20,  0,   0,   0,   228, 228, 228, 0,   0,   0,   0,   0,   0,   0,   0,   20,  183, 20,  0,   0,   228, 228, 228, 228, 0,   0,   0,   0,   20,  183, 183, 20,  0,   0,   0,   0,   228, 228, 228, 228, 80,  80,  80,  80,  80,  80,  80,  152, 152, 80,  80,  80,  228, 228, 228, 80,  80,  80,  80,  80,  80,  80,  80,  152, 253, 152, 80,  80,  228, 228, 228, 228, 80,  80,  80,  80,  152, 253, 253, 152, 80,  80,  80,  80,  228, 228, 228, 228, 0,   0,   0,   0,   0,   0,   0,   173, 173, 0,   0,   0,   228, 228, 228, 0,   0,   0,   0,   0,   0,   0,   0,   173, 255, 173, 0,   0,   228, 228, 228, 228, 0,   0,   0,   0,   173, 255, 255, 173, 0,   0,   0,   0,   228, 228, 228, 228, 38,  38,  13,  255, 255, 13,  13,  13,  13,  13,  38,  38,  228, 228, 
	228, 0,   0,   0,   0,   0,   0,   0,   0,   20,  183, 20,  0,   0,   228, 228, 228, 0,   0,   0,   0,   0,   0,   0,   0,   0,   20,  20,  0,   0,   228, 228, 228, 228, 0,   0,   0,   0,   0,   20,  20,  0,   0,   0,   0,   0,   228, 228, 228, 80,  80,  80,  80,  80,  80,  80,  80,  152, 253, 152, 80,  80,  228, 228, 228, 80,  80,  80,  80,  80,  80,  80,  80,  80,  152, 152, 80,  80,  228, 228, 228, 228, 80,  80,  80,  80,  80,  152, 152, 80,  80,  80,  80,  80,  228, 228, 228, 0,   0,   0,   0,   0,   0,   0,   0,   173, 255, 173, 0,   0,   228, 228, 228, 0,   0,   0,   0,   0,   0,   0,   0,   0,   173, 173, 0,   0,   228, 228, 228, 228, 0,   0,   0,   0,   0,   173, 173, 0,   0,   0,   0,   0,   228, 228, 228, 228, 38,  13,  255, 255, 13,  13,  13,  13,  13,  13,  13,  38,  228, 228, 
	228, 0,   0,   0,   0,   0,   0,   0,   0,   0,   20,  20,  0,   0,   228, 228, 20,  20,  20,  20,  0,   0,   0,   0,   0,   0,   0,   0,   0,   0,   0,   228, 228, 228, 0,   0,   0,   0,   0,   0,   0,   0,   0,   0,   0,   0,   228, 228, 228, 80,  80,  80,  80,  80,  80,  80,  80,  80,  152, 152, 80,  80,  228, 228, 152, 152, 152, 152, 80,  80,  80,  80,  80,  80,  80,  80,  80,  80,  80,  228, 228, 228, 80,  80,  80,  80,  80,  80,  80,  80,  80,  80,  80,  80,  228, 228, 228, 0,   0,   0,   0,   0,   0,   0,   0,   0,   173, 173, 0,   0,   228, 228, 173, 173, 173, 173, 0,   0,   0,   0,   0,   0,   0,   0,   0,   0,   0,   228, 228, 228, 0,   0,   0,   0,   0,   0,   0,   0,   0,   0,   0,   0,   228, 228, 228, 228, 13,  13,  255, 13,  13,  13,  13,  13,  13,  13,  13,  13,  228, 228, 
	20,  20,  20,  20,  0,   0,   0,   0,   0,   0,   0,   0,   0,   0,   0,   228, 228, 0,   0,   20,  20,  0,   0,   0,   0,   0,   0,   0,   0,   0,   0,   228, 228, 0,   0,   0,   0,   0,   0,   0,   0,   0,   0,   0,   0,   0,   0,   228, 152, 152, 152, 152, 80,  80,  80,  80,  80,  80,  80,  80,  80,  80,  80,  228, 228, 80,  80,  152, 152, 80,  80,  80,  80,  80,  80,  80,  80,  80,  80,  228, 228, 80,  80,  80,  80,  80,  80,  80,  80,  80,  80,  80,  80,  80,  80,  228, 173, 173, 173, 173, 0,   0,   0,   0,   0,   0,   0,   0,   0,   0,   0,   228, 228, 0,   0,   173, 173, 0,   0,   0,   0,   0,   0,   0,   0,   0,   0,   228, 228, 0,   0,   0,   0,   0,   0,   0,   0,   0,   0,   0,   0,   0,   0,   228, 228, 38,  38,  13,  13,  13,  13,  13,  13,  13,  13,  13,  13,  38,  38,  228, 
	228, 0,   0,   20,  20,  0,   0,   0,   0,   0,   0,   0,   0,   0,   0,   228, 228, 0,   0,   0,   20,  0,   0,   0,   0,   0,   0,   0,   0,   0,   0,   228, 228, 0,   0,   0,   0,   0,   0,   0,   0,   0,   0,   0,   0,   0,   0,   228, 228, 80,  80,  152, 152, 80,  80,  80,  80,  80,  80,  80,  80,  80,  80,  228, 228, 80,  80,  80,  152, 80,  80,  80,  80,  80,  80,  80,  80,  80,  80,  228, 228, 80,  80,  80,  80,  80,  80,  80,  80,  80,  80,  80,  80,  80,  80,  228, 228, 0,   0,   173, 173, 0,   0,   0,   0,   0,   0,   0,   0,   0,   0,   228, 228, 0,   0,   0,   173, 0,   0,   0,   0,   0,   0,   0,   0,   0,   0,   228, 228, 0,   0,   0,   0,   0,   0,   0,   0,   0,   0,   0,   0,   0,   0,   228, 228, 38,  38,  13,  13,  13,  13,  13,  13,  13,  13,  13,  13,  38,  38,  228, 
	228, 0,   0,   0,   20,  0,   0,   0,   0,   0,   0,   0,   0,   0,   0,   228, 0,   0,   183, 0,   20,  0,   0,   0,   0,   0,   0,   0,   0,   0,   0,   228, 228, 0,   0,   0,   0,   0,   0,   0,   0,   0,   0,   0,   0,   0,   0,   228, 228, 80,  80,  80,  152, 80,  80,  80,  80,  80,  80,  80,  80,  80,  80,  228, 80,  80,  253, 80,  152, 80,  80,  80,  80,  80,  80,  80,  80,  80,  80,  228, 228, 80,  80,  80,  80,  80,  80,  80,  80,  80,  80,  80,  80,  80,  80,  228, 228, 0,   0,   0,   173, 0,   0,   0,   0,   0,   0,   0,   0,   0,   0,   228, 0,   0,   255, 0,   173, 0,   0,   0,   0,   0,   0,   0,   0,   0,   0,   228, 228, 0,   0,   0,   0,   0,   0,   0,   0,   0,   0,   0,   0,   0,   0,   228, 228, 228, 13,  13,  13,  13,  13,  13,  13,  13,  13,  255, 13,  13,  228, 228, 
	0,   0,   183, 0,   20,  0,   0,   0,   0,   0,   0,   0,   0,   0,   0,   228, 0,   0,   0,   0,   20,  0,   0,   0,   0,   0,   0,   0,   0,   0,   0,   228, 228, 0,   0,   0,   0,   0,   0,   0,   0,   0,   0,   0,   0,   0,   0,   228, 80,  80,  253, 80,  152, 80,  80,  80,  80,  80,  80,  80,  80,  80,  80,  228, 80,  80,  80,  80,  152, 80,  80,  80,  80,  80,  80,  80,  80,  80,  80,  228, 228, 80,  80,  80,  80,  80,  80,  80,  80,  80,  80,  80,  80,  80,  80,  228, 0,   0,   255, 0,   173, 0,   0,   0,   0,   0,   0,   0,   0,   0,   0,   228, 0,   0,   0,   0,   173, 0,   0,   0,   0,   0,   0,   0,   0,   0,   0,   228, 228, 0,   0,   0,   0,   0,   0,   0,   0,   0,   0,   0,   0,   0,   0,   228, 228, 228, 38,  13,  13,  13,  13,  13,  13,  13,  255, 255, 13,  38,  228, 228, 
	0,   0,   0,   0,   20,  0,   0,   0,   0,   0,   0,   0,   0,   0,   0,   228, 0,   0,   0,   0,   20,  0,   0,   0,   0,   0,   0,   0,   0,   0,   0,   228, 228, 0,   0,   0,   0,   0,   0,   0,   0,   0,   0,   0,   0,   0,   0,   228, 80,  80,  80,  80,  152, 80,  80,  80,  80,  80,  80,  80,  80,  80,  80,  228, 80,  80,  80,  80,  152, 80,  80,  80,  80,  80,  80,  80,  80,  80,  80,  228, 228, 80,  80,  80,  80,  80,  80,  80,  80,  80,  80,  80,  80,  80,  80,  228, 0,   0,   0,   0,   173, 0,   0,   0,   0,   0,   0,   0,   0,   0,   0,   228, 0,   0,   0,   0,   173, 0,   0,   0,   0,   0,   0,   0,   0,   0,   0,   228, 228, 0,   0,   0,   0,   0,   0,   0,   0,   0,   0,   0,   0,   0,   0,   228, 228, 228, 38,  38,  13,  13,  13,  13,  13,  255, 255, 13,  38,  38,  228, 228, 
	0,   0,   0,   0,   20,  0,   0,   0,   0,   0,   0,   0,   0,   0,   0,   228, 228, 0,   0,   183, 20,  20,  0,   0,   0,   0,   0,   20,  20,  20,  20,  20,  228, 0,   0,   0,   0,   0,   0,   0,   0,   0,   0,   0,   0,   0,   0,   228, 80,  80,  80,  80,  152, 80,  80,  80,  80,  80,  80,  80,  80,  80,  80,  228, 228, 80,  80,  253, 152, 152, 80,  80,  80,  80,  80,  152, 152, 152, 152, 152, 228, 80,  80,  80,  80,  80,  80,  80,  80,  80,  80,  80,  80,  80,  80,  228, 0,   0,   0,   0,   173, 0,   0,   0,   0,   0,   0,   0,   0,   0,   0,   228, 228, 0,   0,   255, 173, 173, 0,   0,   0,   0,   0,   173, 173, 173, 173, 173, 228, 0,   0,   0,   0,   0,   0,   0,   0,   0,   0,   0,   0,   0,   0,   228, 228, 228, 228, 13,  13,  13,  13,  13,  13,  13,  13,  13,  13,  228, 228, 228, 
	228, 0,   0,   183, 20,  20,  0,   0,   0,   0,   0,   20,  20,  20,  20,  20,  228, 228, 228, 183, 183, 20,  20,  0,   0,   20,  20,  20,  183, 183, 228, 228, 20,  20,  20,  20,  20,  0,   0,   0,   0,   0,   0,   20,  20,  20,  20,  20,  228, 80,  80,  253, 152, 152, 80,  80,  80,  80,  80,  152, 152, 152, 152, 152, 228, 228, 228, 253, 253, 152, 152, 80,  80,  152, 152, 152, 253, 253, 228, 228, 152, 152, 152, 152, 152, 80,  80,  80,  80,  80,  80,  152, 152, 152, 152, 152, 228, 0,   0,   255, 173, 173, 0,   0,   0,   0,   0,   173, 173, 173, 173, 173, 228, 228, 228, 255, 255, 173, 173, 0,   0,   173, 173, 173, 255, 255, 228, 228, 173, 173, 173, 173, 173, 0,   0,   0,   0,   0,   0,   173, 173, 173, 173, 173, 228, 228, 228, 228, 38,  13,  13,  13,  13,  13,  13,  38,  228, 228, 228, 228, 
	228, 228, 183, 183, 183, 20,  20,  0,   0,   20,  20,  20,  183, 183, 183, 228, 228, 228, 228, 183, 183, 183, 20,  20,  20,  20,  228, 183, 183, 183, 228, 228, 228, 228, 228, 228, 20,  20,  20,  0,   0,   20,  20,  20,  228, 228, 228, 228, 228, 228, 253, 253, 253, 152, 152, 80,  80,  152, 152, 152, 253, 253, 253, 228, 228, 228, 228, 253, 253, 253, 152, 152, 152, 152, 228, 253, 253, 253, 228, 228, 228, 228, 228, 228, 152, 152, 152, 80,  80,  152, 152, 152, 228, 228, 228, 228, 228, 228, 255, 255, 255, 173, 173, 0,   0,   173, 173, 173, 255, 255, 255, 228, 228, 228, 228, 255, 255, 255, 173, 173, 173, 173, 228, 255, 255, 255, 228, 228, 228, 228, 228, 228, 173, 173, 173, 0,   0,   173, 173, 173, 228, 228, 228, 228, 228, 228, 228, 228, 38,  38,  13,  38,  38,  13,  38,  38,  228, 228, 228, 228, 
	228, 183, 183, 183, 183, 228, 20,  20,  20,  20,  228, 228, 183, 183, 183, 183, 228, 228, 228, 228, 183, 183, 228, 228, 228, 228, 228, 183, 183, 228, 228, 228, 228, 228, 228, 228, 228, 228, 20,  20,  20,  20,  228, 228, 228, 228, 228, 228, 228, 253, 253, 253, 253, 228, 152, 152, 152, 152, 228, 228, 253, 253, 253, 253, 228, 228, 228, 228, 253, 253, 228, 228, 228, 228, 228, 253, 253, 228, 228, 228, 228, 228, 228, 228, 228, 228, 152, 152, 152, 152, 228, 228, 228, 228, 228, 228, 228, 136, 255, 255, 255, 228, 173, 173, 173, 173, 228, 228, 255, 255, 255, 255, 228, 228, 228, 228, 255, 255, 228, 228, 228, 228, 228, 255, 255, 228, 228, 228, 228, 228, 228, 228, 228, 228, 173, 173, 173, 173, 228, 228, 228, 228, 228, 228, 228, 228, 228, 228, 228, 228, 228, 38,  38,  228, 228, 228, 228, 228, 228, 228 
}

\def\arrayEnemiesThree{
	228, 228, 228, 228, 228, 228, 228, 228, 228, 228, 228, 228, 255, 255, 255, 228, 228, 228, 228, 228, 228, 228, 228, 228, 228, 228, 228, 228, 228, 228, 228, 228, 228, 228, 228, 228, 228, 38,  38,  38,  38,  38,  228, 228, 228, 228, 228, 228, 228, 228, 228, 228, 228, 228, 228, 228, 228, 228, 228, 228, 183, 183, 183, 228, 228, 228, 228, 228, 228, 228, 228, 228, 228, 228, 228, 228, 228, 228, 228, 228, 228, 228, 228, 228, 228, 20,  20,  20,  20,  20,  228, 228, 228, 228, 228, 228, 228, 228, 228, 228, 228, 228, 228, 228, 228, 228, 228, 228, 255, 255, 255, 228, 228, 228, 228, 228, 228, 228, 228, 228, 228, 228, 228, 228, 228, 228, 228, 228, 228, 228, 228, 228, 228, 38,  38,  38,  38,  38,  228, 228, 228, 228, 228, 228, 228, 228, 228, 228, 228, 228, 228, 228, 228, 228, 228, 228, 228, 228, 228, 228, 
	228, 228, 228, 228, 228, 228, 228, 228, 32,  32,  255, 255, 255, 38,  228, 228, 228, 228, 228, 228, 228, 228, 228, 228, 228, 228, 228, 228, 228, 228, 228, 228, 228, 228, 228, 228, 38,  38,  38,  38,  38,  38,  38,  228, 228, 228, 228, 228, 228, 228, 228, 228, 228, 228, 228, 228, 152, 152, 183, 183, 183, 20,  228, 228, 228, 228, 228, 228, 228, 228, 228, 228, 228, 228, 228, 228, 228, 228, 228, 228, 228, 228, 228, 228, 20,  20,  20,  20,  20,  20,  20,  228, 228, 228, 228, 228, 228, 228, 228, 228, 228, 228, 228, 228, 173, 173, 255, 255, 255, 38,  228, 228, 228, 228, 228, 228, 228, 228, 228, 228, 228, 228, 228, 228, 228, 228, 228, 228, 228, 228, 228, 228, 38,  38,  38,  38,  38,  38,  38,  228, 228, 228, 228, 228, 228, 228, 228, 228, 228, 228, 228, 228, 228, 228, 228, 228, 228, 228, 228, 228, 
	228, 228, 228, 228, 228, 228, 228, 32,  32,  32,  255, 255, 38,  38,  228, 228, 228, 228, 228, 228, 228, 228, 228, 228, 228, 228, 228, 228, 228, 228, 228, 228, 228, 228, 228, 32,  32,  32,  38,  32,  32,  32,  38,  38,  228, 228, 228, 228, 228, 228, 228, 228, 228, 228, 228, 152, 152, 152, 183, 183, 20,  20,  228, 228, 228, 228, 228, 228, 228, 228, 228, 228, 228, 228, 228, 228, 228, 228, 228, 228, 228, 228, 228, 152, 152, 152, 20,  152, 152, 152, 20,  20,  228, 228, 228, 228, 228, 228, 228, 228, 228, 228, 228, 173, 173, 173, 255, 255, 38,  38,  228, 228, 228, 228, 228, 228, 228, 228, 228, 228, 228, 228, 228, 228, 228, 228, 228, 228, 228, 228, 228, 173, 173, 173, 38,  173, 173, 173, 38,  38,  228, 228, 228, 228, 228, 228, 228, 228, 228, 228, 228, 228, 228, 228, 228, 228, 228, 228, 228, 228, 
	228, 228, 228, 228, 228, 255, 255, 32,  32,  32,  32,  38,  38,  32,  228, 228, 228, 228, 228, 228, 228, 228, 228, 228, 228, 228, 228, 228, 228, 228, 228, 228, 228, 228, 32,  255, 255, 255, 32,  255, 255, 255, 32,  38,  228, 228, 228, 228, 228, 228, 228, 228, 228, 183, 183, 152, 152, 152, 152, 20,  20,  152, 228, 228, 228, 228, 228, 228, 228, 228, 228, 228, 228, 228, 228, 228, 228, 228, 228, 228, 228, 228, 152, 183, 183, 183, 152, 183, 183, 183, 152, 20,  228, 228, 228, 228, 228, 228, 228, 228, 228, 255, 255, 173, 173, 173, 173, 38,  38,  173, 228, 228, 228, 228, 228, 228, 228, 228, 228, 228, 228, 228, 228, 228, 228, 228, 228, 228, 228, 228, 173, 255, 255, 255, 173, 255, 255, 255, 173, 38,  228, 228, 228, 228, 228, 228, 228, 228, 228, 228, 228, 228, 228, 228, 228, 228, 228, 228, 228, 228, 
	228, 38,  228, 228, 32,  255, 255, 32,  32,  32,  32,  32,  32,  32,  228, 228, 228, 228, 228, 228, 228, 228, 228, 228, 228, 228, 228, 228, 228, 228, 228, 228, 228, 228, 32,  255, 255, 255, 255, 255, 255, 255, 32,  38,  32,  228, 228, 228, 228, 20,  228, 228, 152, 183, 183, 152, 152, 152, 152, 152, 152, 152, 228, 228, 228, 228, 228, 228, 228, 228, 228, 228, 228, 228, 228, 228, 228, 228, 228, 228, 228, 228, 152, 183, 183, 183, 183, 183, 183, 183, 152, 20,  152, 228, 228, 228, 228, 38,  228, 228, 173, 255, 255, 173, 173, 173, 173, 173, 173, 173, 228, 228, 228, 228, 228, 228, 228, 228, 228, 228, 228, 228, 228, 228, 228, 228, 228, 228, 228, 228, 173, 255, 255, 255, 255, 255, 255, 255, 173, 38,  173, 228, 228, 228, 228, 228, 228, 228, 228, 228, 228, 228, 228, 228, 228, 228, 228, 228, 228, 228, 
	38,  228, 38,  32,  32,  255, 255, 32,  32,  32,  32,  32,  32,  32,  32,  228, 228, 228, 228, 228, 228, 228, 228, 228, 228, 228, 228, 228, 228, 228, 228, 228, 228, 228, 32,  255, 255, 32,  255, 32,  255, 255, 32,  38,  32,  32,  228, 228, 20,  228, 20,  152, 152, 183, 183, 152, 152, 152, 152, 152, 152, 152, 152, 228, 228, 228, 228, 228, 228, 228, 228, 228, 228, 228, 228, 228, 228, 228, 228, 228, 228, 228, 152, 183, 183, 152, 183, 152, 183, 183, 152, 20,  152, 152, 228, 228, 38,  228, 38,  173, 173, 255, 255, 173, 173, 173, 173, 173, 173, 173, 173, 228, 228, 228, 228, 228, 228, 228, 228, 228, 228, 228, 228, 228, 228, 228, 228, 228, 228, 228, 173, 255, 255, 173, 255, 173, 255, 255, 173, 38,  173, 173, 228, 228, 228, 228, 228, 228, 228, 228, 228, 228, 228, 228, 228, 228, 228, 228, 228, 228, 
	38,  38,  38,  255, 255, 255, 32,  32,  32,  38,  32,  32,  32,  32,  32,  228, 228, 228, 228, 228, 228, 228, 228, 228, 228, 228, 228, 228, 228, 228, 228, 228, 228, 228, 32,  255, 255, 32,  255, 32,  255, 255, 32,  38,  32,  32,  228, 228, 20,  20,  20,  183, 183, 183, 152, 152, 152, 20,  152, 152, 152, 152, 152, 228, 228, 228, 228, 228, 228, 228, 228, 228, 228, 228, 228, 228, 228, 228, 228, 228, 228, 228, 152, 183, 183, 152, 183, 152, 183, 183, 152, 20,  152, 152, 228, 228, 38,  38,  38,  255, 255, 255, 173, 173, 173, 38,  173, 173, 173, 173, 173, 228, 228, 228, 228, 228, 228, 228, 228, 228, 228, 228, 228, 228, 228, 228, 228, 228, 228, 228, 173, 255, 255, 173, 255, 173, 255, 255, 173, 38,  173, 173, 228, 228, 228, 228, 228, 228, 228, 228, 228, 228, 228, 228, 228, 228, 228, 228, 228, 228, 
	38,  38,  38,  38,  255, 32,  32,  32,  38,  38,  38,  32,  32,  32,  32,  228, 228, 228, 228, 228, 228, 228, 228, 228, 228, 228, 228, 228, 228, 228, 228, 228, 228, 228, 228, 32,  32,  32,  38,  32,  32,  32,  32,  32,  32,  32,  228, 228, 20,  20,  20,  20,  183, 152, 152, 152, 20,  20,  20,  152, 152, 152, 152, 228, 228, 228, 228, 228, 228, 228, 228, 228, 228, 228, 228, 228, 228, 228, 228, 228, 228, 228, 228, 152, 152, 152, 20,  152, 152, 152, 152, 152, 152, 152, 228, 228, 38,  38,  38,  38,  255, 173, 173, 173, 38,  38,  38,  173, 173, 173, 173, 228, 228, 228, 228, 228, 228, 228, 228, 228, 228, 228, 228, 228, 228, 228, 228, 228, 228, 228, 228, 173, 173, 173, 38,  173, 173, 173, 173, 173, 173, 173, 228, 228, 228, 228, 228, 228, 228, 228, 228, 228, 228, 228, 228, 228, 228, 228, 228, 228, 
	255, 38,  38,  38,  32,  32,  38,  38,  255, 32,  38,  32,  32,  32,  32,  255, 255, 32,  32,  32,  255, 228, 228, 228, 228, 228, 228, 228, 228, 228, 228, 228, 228, 228, 228, 38,  38,  38,  32,  32,  32,  32,  38,  38,  38,  32,  32,  228, 183, 20,  20,  20,  152, 152, 20,  20,  183, 152, 20,  152, 152, 152, 152, 183, 183, 152, 152, 152, 183, 228, 228, 228, 228, 228, 228, 228, 228, 228, 228, 228, 228, 228, 228, 20,  20,  20,  152, 152, 152, 152, 20,  20,  20,  152, 152, 228, 255, 38,  38,  38,  173, 173, 38,  38,  255, 173, 38,  173, 173, 173, 173, 255, 255, 173, 173, 173, 255, 228, 228, 228, 228, 228, 228, 228, 228, 228, 228, 228, 228, 228, 228, 38,  38,  38,  173, 173, 173, 173, 38,  38,  38,  173, 173, 228, 228, 228, 228, 228, 228, 228, 228, 228, 228, 228, 228, 228, 228, 228, 228, 228, 
	228, 255, 228, 38,  38,  38,  38,  32,  32,  32,  38,  32,  32,  32,  32,  255, 255, 32,  32,  255, 255, 38,  32,  228, 228, 228, 228, 228, 228, 228, 228, 228, 228, 228, 38,  38,  38,  38,  38,  32,  32,  38,  38,  38,  38,  38,  32,  228, 228, 183, 228, 20,  20,  20,  20,  152, 152, 152, 20,  152, 152, 152, 152, 183, 183, 152, 152, 183, 183, 20,  152, 228, 228, 228, 228, 228, 228, 228, 228, 228, 228, 228, 20,  20,  20,  20,  20,  152, 152, 20,  20,  20,  20,  20,  152, 228, 228, 255, 228, 38,  38,  38,  38,  173, 173, 173, 38,  173, 173, 173, 173, 255, 255, 173, 173, 255, 255, 38,  173, 228, 228, 228, 228, 228, 228, 228, 228, 228, 228, 228, 38,  38,  38,  38,  38,  173, 173, 38,  38,  38,  38,  38,  173, 228, 228, 228, 13,  228, 228, 228, 228, 228, 228, 228, 228, 228, 228, 13,  228, 228, 
	228, 255, 228, 255, 255, 32,  255, 32,  32,  255, 38,  32,  32,  32,  32,  255, 255, 32,  255, 255, 255, 38,  38,  32,  255, 255, 255, 228, 228, 228, 228, 228, 228, 32,  38,  38,  38,  38,  38,  255, 255, 38,  38,  38,  38,  38,  32,  228, 228, 183, 228, 183, 183, 152, 183, 152, 152, 183, 20,  152, 152, 152, 152, 183, 183, 152, 183, 183, 183, 20,  20,  152, 183, 183, 183, 228, 228, 228, 228, 228, 228, 152, 20,  20,  20,  20,  20,  183, 183, 20,  20,  20,  20,  20,  152, 228, 228, 255, 228, 255, 255, 173, 255, 173, 173, 255, 38,  173, 173, 173, 173, 255, 255, 173, 255, 255, 255, 38,  38,  173, 255, 255, 255, 228, 228, 228, 228, 228, 228, 173, 38,  38,  38,  38,  38,  255, 255, 38,  38,  38,  38,  38,  173, 228, 228, 13,  38,  255, 255, 228, 228, 228, 228, 228, 228, 255, 255, 38,  13,  228, 
	228, 228, 228, 228, 255, 228, 228, 228, 32,  38,  38,  32,  32,  32,  255, 255, 255, 32,  32,  255, 38,  38,  32,  32,  255, 255, 38,  228, 228, 228, 228, 228, 228, 32,  255, 38,  38,  38,  255, 255, 255, 255, 38,  38,  38,  255, 32,  228, 228, 228, 228, 228, 183, 228, 228, 228, 152, 20,  20,  152, 152, 152, 183, 183, 183, 152, 152, 183, 20,  20,  152, 152, 183, 183, 20,  228, 228, 228, 228, 228, 228, 152, 183, 20,  20,  20,  183, 183, 183, 183, 20,  20,  20,  183, 152, 228, 228, 228, 228, 228, 255, 228, 228, 228, 173, 38,  38,  173, 173, 173, 255, 255, 255, 173, 173, 255, 38,  38,  173, 173, 255, 255, 38,  228, 228, 228, 228, 228, 228, 173, 255, 38,  38,  38,  255, 255, 255, 255, 38,  38,  38,  255, 173, 228, 228, 13,  13,  255, 228, 228, 228, 228, 228, 228, 228, 228, 255, 13,  13,  228, 
	228, 228, 228, 228, 228, 228, 228, 228, 255, 38,  38,  32,  32,  255, 255, 255, 32,  32,  32,  32,  32,  32,  32,  32,  32,  38,  38,  32,  255, 228, 228, 228, 228, 32,  255, 255, 255, 255, 255, 255, 255, 255, 255, 255, 255, 255, 32,  228, 228, 228, 228, 228, 228, 228, 228, 228, 183, 20,  20,  152, 152, 183, 183, 183, 152, 152, 152, 152, 152, 152, 152, 152, 152, 20,  20,  152, 183, 228, 228, 228, 228, 152, 183, 183, 183, 183, 183, 183, 183, 183, 183, 183, 183, 183, 152, 228, 228, 228, 228, 228, 228, 228, 228, 228, 255, 38,  38,  173, 173, 255, 255, 255, 173, 173, 173, 173, 173, 173, 173, 173, 173, 38,  38,  173, 255, 228, 228, 228, 228, 173, 255, 255, 255, 255, 255, 255, 255, 255, 255, 255, 255, 255, 173, 228, 228, 38,  13,  13,  255, 255, 228, 228, 228, 228, 255, 255, 13,  13,  38,  228, 
	228, 228, 228, 228, 228, 228, 228, 228, 38,  38,  32,  32,  255, 255, 255, 32,  32,  32,  32,  32,  32,  32,  32,  32,  32,  32,  32,  32,  255, 228, 228, 228, 32,  255, 255, 255, 255, 255, 255, 255, 255, 255, 255, 255, 255, 255, 255, 32,  228, 228, 228, 228, 228, 228, 228, 228, 20,  20,  152, 152, 183, 183, 183, 152, 152, 152, 152, 152, 152, 152, 152, 152, 152, 152, 152, 152, 183, 228, 228, 228, 152, 183, 183, 183, 183, 183, 183, 183, 183, 183, 183, 183, 183, 183, 183, 152, 228, 228, 228, 228, 228, 228, 228, 228, 38,  38,  173, 173, 255, 255, 255, 173, 173, 173, 173, 173, 173, 173, 173, 173, 173, 173, 173, 173, 255, 228, 228, 228, 173, 255, 255, 255, 255, 255, 255, 255, 255, 255, 255, 255, 255, 255, 255, 173, 13,  13,  13,  38,  255, 228, 228, 228, 228, 228, 228, 255, 38,  13,  13,  13,  
	228, 228, 228, 228, 228, 228, 255, 38,  38,  38,  32,  32,  255, 255, 255, 32,  32,  32,  32,  32,  32,  255, 255, 255, 32,  32,  32,  32,  32,  228, 228, 228, 32,  255, 255, 255, 255, 255, 32,  255, 255, 32,  255, 255, 255, 255, 255, 32,  228, 228, 228, 228, 228, 228, 183, 20,  20,  20,  152, 152, 183, 183, 183, 152, 152, 152, 152, 152, 152, 183, 183, 183, 152, 152, 152, 152, 152, 228, 228, 228, 152, 183, 183, 183, 183, 183, 152, 183, 183, 152, 183, 183, 183, 183, 183, 152, 228, 228, 228, 228, 228, 228, 255, 38,  38,  38,  173, 173, 255, 255, 255, 173, 173, 173, 173, 173, 173, 255, 255, 255, 173, 173, 173, 173, 173, 228, 228, 228, 173, 255, 255, 255, 255, 255, 173, 255, 255, 173, 255, 255, 255, 255, 255, 173, 38,  13,  13,  13,  13,  255, 255, 228, 228, 255, 255, 13,  13,  13,  13,  38,  
	228, 228, 228, 228, 228, 228, 228, 38,  38,  228, 228, 32,  32,  255, 255, 255, 255, 255, 255, 32,  32,  255, 255, 38,  32,  32,  32,  255, 255, 255, 228, 228, 32,  255, 255, 255, 255, 255, 32,  255, 255, 32,  255, 255, 255, 255, 255, 32,  228, 228, 228, 228, 228, 228, 228, 20,  20,  228, 228, 152, 152, 183, 183, 183, 183, 183, 183, 152, 152, 183, 183, 20,  152, 152, 152, 183, 183, 183, 228, 228, 152, 183, 183, 183, 183, 183, 152, 183, 183, 152, 183, 183, 183, 183, 183, 152, 228, 228, 228, 228, 228, 228, 228, 38,  38,  228, 228, 173, 173, 255, 255, 255, 255, 255, 255, 173, 173, 255, 255, 38,  173, 173, 173, 255, 255, 255, 228, 228, 173, 255, 255, 255, 255, 255, 173, 255, 255, 173, 255, 255, 255, 255, 255, 173, 13,  13,  38,  13,  13,  255, 228, 228, 228, 228, 255, 13,  13,  38,  13,  13,  
	228, 228, 228, 228, 228, 228, 228, 228, 228, 228, 228, 228, 228, 32,  32,  38,  38,  38,  255, 255, 32,  32,  38,  38,  32,  32,  32,  255, 255, 38,  228, 228, 32,  255, 255, 255, 255, 255, 32,  255, 255, 32,  255, 255, 255, 255, 255, 32,  228, 228, 228, 228, 228, 228, 228, 228, 228, 228, 228, 228, 228, 152, 152, 20,  20,  20,  183, 183, 152, 152, 20,  20,  152, 152, 152, 183, 183, 20,  228, 228, 152, 183, 183, 183, 183, 183, 152, 183, 183, 152, 183, 183, 183, 183, 183, 152, 228, 228, 228, 228, 228, 228, 228, 228, 228, 228, 228, 228, 228, 173, 173, 38,  38,  38,  255, 255, 173, 173, 38,  38,  173, 173, 173, 255, 255, 38,  228, 228, 173, 255, 255, 255, 255, 255, 173, 255, 255, 173, 255, 255, 255, 255, 255, 173, 13,  13,  13,  13,  13,  228, 228, 228, 228, 228, 228, 13,  13,  13,  13,  13,  
	228, 228, 228, 228, 228, 228, 228, 228, 228, 38,  38,  38,  228, 228, 255, 228, 38,  38,  38,  255, 255, 32,  32,  32,  32,  32,  32,  32,  38,  38,  228, 228, 32,  255, 255, 255, 255, 255, 255, 255, 255, 255, 255, 255, 255, 255, 255, 32,  228, 228, 228, 228, 228, 228, 228, 228, 228, 20,  20,  20,  228, 228, 183, 228, 20,  20,  20,  183, 183, 152, 152, 152, 152, 152, 152, 152, 20,  20,  228, 228, 152, 183, 183, 183, 183, 183, 183, 183, 183, 183, 183, 183, 183, 183, 183, 152, 228, 228, 228, 228, 228, 228, 228, 228, 228, 38,  38,  38,  228, 228, 255, 228, 38,  38,  38,  255, 255, 173, 173, 173, 173, 173, 173, 173, 38,  38,  228, 228, 173, 255, 255, 255, 255, 255, 255, 255, 255, 255, 255, 255, 255, 255, 255, 173, 13,  38,  13,  13,  13,  13,  255, 228, 228, 255, 13,  13,  13,  13,  38,  13,  
	228, 228, 228, 228, 228, 228, 228, 228, 38,  38,  255, 228, 38,  38,  38,  228, 255, 38,  38,  32,  255, 32,  32,  32,  32,  32,  32,  32,  32,  32,  255, 228, 32,  255, 32,  255, 255, 255, 255, 255, 255, 255, 255, 255, 255, 32,  255, 32,  228, 228, 228, 228, 228, 228, 228, 228, 20,  20,  183, 228, 20,  20,  20,  228, 183, 20,  20,  152, 183, 152, 152, 152, 152, 152, 152, 152, 152, 152, 183, 228, 152, 183, 152, 183, 183, 183, 183, 183, 183, 183, 183, 183, 183, 152, 183, 152, 228, 228, 228, 228, 228, 228, 228, 228, 38,  38,  255, 228, 38,  38,  38,  228, 255, 38,  38,  173, 255, 173, 173, 173, 173, 173, 173, 173, 173, 173, 255, 228, 173, 255, 173, 255, 255, 255, 255, 255, 255, 255, 255, 255, 255, 173, 255, 173, 13,  13,  13,  13,  38,  13,  255, 228, 228, 255, 13,  38,  13,  13,  13,  13,  
	228, 228, 228, 228, 228, 228, 228, 228, 38,  38,  228, 228, 38,  38,  38,  38,  228, 228, 32,  32,  255, 32,  32,  32,  255, 255, 255, 32,  32,  32,  38,  255, 228, 32,  255, 255, 255, 32,  255, 255, 255, 255, 32,  255, 255, 255, 32,  228, 228, 228, 228, 228, 228, 228, 228, 228, 20,  20,  228, 228, 20,  20,  20,  20,  228, 228, 152, 152, 183, 152, 152, 152, 183, 183, 183, 152, 152, 152, 20,  183, 228, 152, 183, 183, 183, 152, 183, 183, 183, 183, 152, 183, 183, 183, 152, 228, 228, 228, 228, 228, 228, 228, 228, 228, 38,  38,  228, 228, 38,  38,  38,  38,  228, 228, 173, 173, 255, 173, 173, 173, 255, 255, 255, 173, 173, 173, 38,  255, 228, 173, 255, 255, 255, 173, 255, 255, 255, 255, 173, 255, 255, 255, 173, 228, 228, 38,  13,  13,  13,  13,  13,  228, 228, 13,  13,  13,  13,  13,  38,  228, 
	228, 228, 228, 228, 228, 228, 228, 228, 38,  255, 228, 38,  38,  38,  38,  38,  38,  255, 32,  32,  255, 255, 32,  32,  255, 255, 38,  32,  32,  32,  228, 228, 228, 32,  255, 255, 255, 255, 32,  32,  32,  32,  255, 255, 255, 255, 32,  228, 228, 228, 228, 228, 228, 228, 228, 228, 20,  183, 228, 20,  20,  20,  20,  20,  20,  183, 152, 152, 183, 183, 152, 152, 183, 183, 20,  152, 152, 152, 228, 228, 228, 152, 183, 183, 183, 183, 152, 152, 152, 152, 183, 183, 183, 183, 152, 228, 228, 228, 228, 228, 228, 228, 228, 228, 38,  255, 228, 38,  38,  38,  38,  38,  38,  255, 173, 173, 255, 255, 173, 173, 255, 255, 38,  173, 173, 173, 228, 228, 228, 173, 255, 255, 255, 255, 173, 173, 173, 173, 255, 255, 255, 255, 173, 228, 228, 13,  13,  38,  13,  13,  13,  228, 228, 13,  13,  13,  38,  13,  13,  228, 
	228, 228, 228, 228, 228, 228, 228, 228, 38,  228, 228, 38,  38,  38,  38,  38,  32,  32,  32,  32,  255, 255, 32,  32,  32,  38,  38,  32,  32,  255, 255, 228, 228, 32,  255, 255, 255, 255, 255, 255, 255, 255, 255, 255, 255, 255, 32,  228, 228, 228, 228, 228, 228, 228, 228, 228, 20,  228, 228, 20,  20,  20,  20,  20,  152, 152, 152, 152, 183, 183, 152, 152, 152, 20,  20,  152, 152, 183, 183, 228, 228, 152, 183, 183, 183, 183, 183, 183, 183, 183, 183, 183, 183, 183, 152, 228, 228, 228, 228, 228, 228, 228, 228, 228, 38,  228, 228, 38,  38,  38,  38,  38,  173, 173, 173, 173, 255, 255, 173, 173, 173, 38,  38,  173, 173, 255, 255, 228, 228, 173, 255, 255, 255, 255, 255, 255, 255, 255, 255, 255, 255, 255, 173, 228, 228, 228, 13,  13,  13,  13,  38,  228, 228, 38,  13,  13,  13,  13,  228, 228, 
	228, 228, 228, 228, 228, 228, 228, 228, 228, 255, 228, 38,  38,  38,  38,  38,  32,  32,  32,  32,  255, 255, 32,  32,  32,  32,  32,  32,  32,  255, 255, 255, 228, 228, 32,  255, 255, 255, 255, 32,  32,  255, 255, 255, 255, 32,  228, 228, 228, 228, 228, 228, 228, 228, 228, 228, 228, 183, 228, 20,  20,  20,  20,  20,  152, 152, 152, 152, 183, 183, 152, 152, 152, 152, 152, 152, 152, 183, 183, 183, 228, 228, 152, 183, 183, 183, 183, 152, 152, 183, 183, 183, 183, 152, 228, 228, 228, 228, 228, 228, 228, 228, 228, 228, 228, 255, 228, 38,  38,  38,  38,  38,  173, 173, 173, 173, 255, 255, 173, 173, 173, 173, 173, 173, 173, 255, 255, 255, 228, 228, 173, 255, 255, 255, 255, 173, 173, 255, 255, 255, 255, 173, 228, 228, 228, 228, 228, 13,  38,  13,  13,  13,  13,  13,  13,  38,  13,  228, 228, 228, 
	228, 228, 228, 228, 228, 228, 228, 228, 228, 228, 228, 255, 38,  38,  38,  228, 32,  32,  32,  32,  32,  255, 32,  32,  32,  32,  255, 255, 32,  38,  38,  228, 228, 228, 228, 32,  32,  32,  32,  228, 228, 32,  32,  32,  32,  228, 228, 228, 228, 228, 228, 228, 228, 228, 228, 228, 228, 228, 228, 183, 20,  20,  20,  228, 152, 152, 152, 152, 152, 183, 152, 152, 152, 152, 183, 183, 152, 20,  20,  228, 228, 228, 228, 152, 152, 152, 152, 228, 228, 152, 152, 152, 152, 228, 228, 228, 228, 228, 228, 228, 228, 228, 228, 228, 228, 228, 228, 255, 38,  38,  38,  228, 173, 173, 173, 173, 173, 255, 173, 173, 173, 173, 255, 255, 173, 38,  38,  228, 228, 228, 228, 173, 173, 173, 173, 228, 228, 173, 173, 173, 173, 228, 228, 228, 228, 228, 228, 228, 228, 13,  38,  13,  13,  38,  13,  228, 228, 228, 228, 228, 
	228, 228, 228, 228, 228, 228, 228, 228, 228, 228, 228, 228, 228, 228, 228, 228, 32,  32,  32,  32,  32,  255, 255, 32,  32,  32,  38,  32,  32,  38,  32,  228, 228, 228, 228, 228, 228, 228, 228, 228, 228, 228, 228, 228, 228, 228, 228, 228, 228, 228, 228, 228, 228, 228, 228, 228, 228, 228, 228, 228, 228, 228, 228, 228, 152, 152, 152, 152, 152, 183, 183, 152, 152, 152, 20,  152, 152, 20,  152, 228, 228, 228, 228, 228, 228, 228, 228, 228, 228, 228, 228, 228, 228, 228, 228, 228, 228, 228, 228, 228, 228, 228, 228, 228, 228, 228, 228, 228, 228, 228, 228, 228, 173, 173, 173, 173, 173, 255, 255, 173, 173, 173, 38,  173, 173, 38,  173, 228, 228, 228, 228, 228, 228, 228, 228, 228, 228, 228, 228, 228, 228, 228, 228, 228, 38,  38,  228, 228, 228, 228, 228, 38,  38,  228, 228, 228, 228, 228, 38,  38,  
	228, 228, 228, 228, 228, 228, 228, 228, 228, 228, 228, 228, 228, 228, 228, 228, 228, 32,  32,  32,  32,  32,  255, 255, 255, 32,  32,  32,  32,  32,  32,  228, 228, 228, 228, 228, 228, 228, 228, 228, 228, 228, 228, 228, 228, 228, 228, 228, 228, 228, 228, 228, 228, 228, 228, 228, 228, 228, 228, 228, 228, 228, 228, 228, 228, 152, 152, 152, 152, 152, 183, 183, 183, 152, 152, 152, 152, 152, 152, 228, 228, 228, 228, 228, 228, 228, 228, 228, 228, 228, 228, 228, 228, 228, 228, 228, 228, 228, 228, 228, 228, 228, 228, 228, 228, 228, 228, 228, 228, 228, 228, 228, 228, 173, 173, 173, 173, 173, 255, 255, 255, 173, 173, 173, 173, 173, 173, 228, 228, 228, 228, 228, 228, 228, 228, 228, 228, 228, 228, 228, 228, 228, 228, 228, 38,  13,  38,  38,  228, 228, 228, 38,  38,  228, 228, 228, 38,  38,  13,  38,  
	228, 228, 228, 228, 228, 228, 228, 228, 228, 228, 228, 228, 228, 228, 228, 228, 228, 228, 32,  32,  32,  32,  32,  255, 255, 255, 255, 32,  32,  32,  32,  228, 228, 228, 228, 228, 228, 228, 228, 228, 228, 228, 228, 228, 228, 228, 228, 228, 228, 228, 228, 228, 228, 228, 228, 228, 228, 228, 228, 228, 228, 228, 228, 228, 228, 228, 152, 152, 152, 152, 152, 183, 183, 183, 183, 152, 152, 152, 152, 228, 228, 228, 228, 228, 228, 228, 228, 228, 228, 228, 228, 228, 228, 228, 228, 228, 228, 228, 228, 228, 228, 228, 228, 228, 228, 228, 228, 228, 228, 228, 228, 228, 228, 228, 173, 173, 173, 173, 173, 255, 255, 255, 255, 173, 173, 173, 173, 228, 228, 228, 228, 228, 228, 228, 228, 228, 228, 228, 228, 228, 228, 228, 228, 228, 228, 38,  13,  38,  38,  228, 228, 38,  38,  228, 228, 38,  38,  13,  38,  228, 
	228, 228, 228, 228, 228, 228, 228, 228, 228, 228, 228, 228, 228, 228, 228, 228, 228, 228, 38,  32,  32,  32,  32,  32,  32,  255, 255, 255, 255, 255, 255, 255, 228, 228, 228, 228, 228, 228, 228, 228, 228, 228, 228, 228, 228, 228, 228, 228, 228, 228, 228, 228, 228, 228, 228, 228, 228, 228, 228, 228, 228, 228, 228, 228, 228, 228, 20,  152, 152, 152, 152, 152, 152, 183, 183, 183, 183, 183, 183, 183, 228, 228, 228, 228, 228, 228, 228, 228, 228, 228, 228, 228, 228, 228, 228, 228, 228, 228, 228, 228, 228, 228, 228, 228, 228, 228, 228, 228, 228, 228, 228, 228, 228, 228, 38,  173, 173, 173, 173, 173, 173, 255, 255, 255, 255, 255, 255, 255, 228, 228, 228, 228, 228, 228, 228, 228, 228, 228, 228, 228, 228, 228, 228, 228, 228, 38,  38,  13,  38,  38,  228, 38,  38,  228, 38,  38,  13,  38,  38,  228, 
	228, 228, 228, 228, 228, 228, 228, 228, 228, 228, 228, 228, 228, 228, 228, 228, 228, 255, 255, 38,  38,  32,  32,  38,  38,  38,  38,  255, 255, 255, 255, 255, 228, 228, 228, 228, 228, 228, 228, 228, 228, 228, 228, 228, 228, 228, 228, 228, 228, 228, 228, 228, 228, 228, 228, 228, 228, 228, 228, 228, 228, 228, 228, 228, 228, 183, 183, 20,  20,  152, 152, 20,  20,  20,  20,  183, 183, 183, 183, 183, 228, 228, 228, 228, 228, 228, 228, 228, 228, 228, 228, 228, 228, 228, 228, 228, 228, 228, 228, 228, 228, 228, 228, 228, 228, 228, 228, 228, 228, 228, 228, 228, 228, 255, 255, 38,  38,  173, 173, 38,  38,  38,  38,  255, 255, 255, 255, 255, 228, 228, 228, 228, 228, 228, 228, 228, 228, 228, 228, 228, 228, 228, 228, 228, 228, 228, 38,  38,  13,  38,  228, 38,  38,  228, 38,  13,  38,  38,  228, 228, 
	228, 228, 228, 228, 228, 228, 228, 228, 228, 228, 228, 228, 228, 228, 228, 228, 255, 255, 255, 38,  38,  38,  38,  38,  38,  38,  38,  38,  38,  255, 255, 228, 228, 228, 228, 228, 228, 228, 228, 228, 228, 228, 228, 228, 228, 228, 228, 228, 228, 228, 228, 228, 228, 228, 228, 228, 228, 228, 228, 228, 228, 228, 228, 228, 183, 183, 183, 20,  20,  20,  20,  20,  20,  20,  20,  20,  20,  183, 183, 228, 228, 228, 228, 228, 228, 228, 228, 228, 228, 228, 228, 228, 228, 228, 228, 228, 228, 228, 228, 228, 228, 228, 228, 228, 228, 228, 228, 228, 228, 228, 228, 228, 255, 255, 255, 38,  38,  38,  38,  38,  38,  38,  38,  38,  38,  255, 255, 228, 228, 228, 228, 228, 228, 228, 228, 228, 228, 228, 228, 228, 228, 228, 228, 228, 228, 228, 38,  38,  38,  13,  38,  38,  38,  38,  13,  38,  38,  38,  228, 228, 
	228, 228, 228, 228, 228, 228, 228, 228, 228, 228, 228, 228, 228, 228, 228, 228, 228, 228, 228, 228, 255, 255, 38,  38,  255, 255, 38,  38,  38,  38,  228, 228, 228, 228, 228, 228, 228, 228, 228, 228, 228, 228, 228, 228, 228, 228, 228, 228, 228, 228, 228, 228, 228, 228, 228, 228, 228, 228, 228, 228, 228, 228, 228, 228, 228, 228, 228, 228, 183, 183, 20,  20,  183, 183, 20,  20,  20,  20,  228, 228, 228, 228, 228, 228, 228, 228, 228, 228, 228, 228, 228, 228, 228, 228, 228, 228, 228, 228, 228, 228, 228, 228, 228, 228, 228, 228, 228, 228, 228, 228, 228, 228, 228, 228, 228, 228, 255, 255, 38,  38,  255, 255, 38,  38,  38,  38,  228, 228, 228, 228, 228, 228, 228, 228, 228, 228, 228, 228, 228, 228, 228, 228, 228, 228, 228, 228, 228, 38,  38,  38,  38,  38,  38,  38,  38,  38,  38,  228, 228, 228, 
	228, 228, 228, 228, 228, 228, 228, 228, 228, 228, 228, 228, 228, 228, 228, 228, 228, 228, 228, 255, 255, 255, 38,  255, 255, 255, 38,  38,  38,  38,  38,  228, 228, 228, 228, 228, 228, 228, 228, 228, 228, 228, 228, 228, 228, 228, 228, 228, 228, 228, 228, 228, 228, 228, 228, 228, 228, 228, 228, 228, 228, 228, 228, 228, 228, 228, 228, 183, 183, 183, 20,  183, 183, 183, 20,  20,  20,  20,  20,  228, 228, 228, 228, 228, 228, 228, 228, 228, 228, 228, 228, 228, 228, 228, 228, 228, 228, 228, 228, 228, 228, 228, 228, 228, 228, 228, 228, 228, 228, 228, 228, 228, 228, 228, 228, 255, 255, 255, 38,  255, 255, 255, 38,  38,  38,  38,  38,  228, 228, 228, 228, 228, 228, 228, 228, 228, 228, 228, 228, 228, 228, 228, 228, 228, 228, 228, 228, 228, 228, 228, 38,  38,  38,  38,  228, 228, 228, 228, 228, 228 
}	% Conjunto de bytes de cor da sprite no tilemap
\def\arrayHalfBrick{	% Conjunto de bytes de cor da sprite no tilemap
	20,  20,  20,  20,  20,  20,  20,  0,   20,  20,  20,  20,  20,  20,  20,  0,   
	20,  20,  20,  20,  20,  20,  20,  0,   20,  20,  20,  20,  20,  20,  20,  0,   
	20,  20,  20,  20,  20,  20,  20,  0,   20,  20,  20,  20,  20,  20,  20,  0,   
	0,   0,   0,   0,   0,   0,   0,   0,   0,   0,   0,   0,   0,   0,   0,   0,   
	20,  20,  20,  0,   20,  20,  20,  20,  20,  20,  20,  0,   20,  20,  20,  20,  
	20,  20,  20,  0,   20,  20,  20,  20,  20,  20,  20,  0,   20,  20,  20,  20,  
	20,  20,  20,  0,   20,  20,  20,  20,  20,  20,  20,  0,   20,  20,  20,  20,  
	0,   0,   0,   0,   0,   0,   0,   0,   0,   0,   0,   0,   0,   0,   0,   0,   
	20,  20,  20,  20,  20,  20,  20,  0,   20,  20,  20,  20,  20,  20,  20,  0,   
	20,  20,  20,  20,  20,  20,  20,  0,   20,  20,  20,  20,  20,  20,  20,  0,   
	20,  20,  20,  20,  20,  20,  20,  0,   20,  20,  20,  20,  20,  20,  20,  0,   
	0,   0,   0,   0,   0,   0,   0,   0,   0,   0,   0,   0,   0,   0,   0,   0,   
	20,  20,  20,  0,   20,  20,  20,  20,  20,  20,  20,  0,   20,  20,  20,  20,  
	20,  20,  20,  0,   20,  20,  20,  20,  20,  20,  20,  0,   20,  20,  20,  20,  
	20,  20,  20,  0,   20,  20,  20,  20,  20,  20,  20,  0,   20,  20,  20,  20,  
	0,   0,   0,   0,   0,   0,   0,   0,   0,   0,   0,   0,   0,   0,   0,   0   
}

\def\arrayHalfEnemiesOne{	% Conjunto de bytes de cor da sprite no tilemap
	20,  20,  228, 228, 228, 228, 228, 228, 228, 228, 228, 228, 228, 228, 20,  20,  20,  20,  228, 228, 228, 228, 228, 228, 228, 228, 228, 228, 228, 228, 228, 228, 228, 228, 228, 228, 228, 228, 228, 228, 228, 228, 228, 228, 228, 228, 152, 152, 152, 152, 228, 228, 228, 228, 228, 228, 228, 228, 228, 228, 228, 228, 152, 152, 152, 152, 228, 228, 228, 228, 228, 228, 228, 228, 228, 228, 228, 228, 228, 228, 228, 228, 228, 228, 228, 228, 228, 228, 228, 228, 228, 228, 228, 228, 173, 173, 173, 173, 228, 228, 228, 228, 228, 228, 228, 228, 228, 228, 228, 228, 173, 173, 173, 173, 228, 228, 228, 228, 228, 228, 228, 228, 228, 228, 228, 228, 228, 228, 228, 228, 228, 228, 228, 228, 228, 228, 228, 228, 228, 228, 228, 228, 228, 228, 228, 228, 228, 228, 228, 228, 228, 228, 228, 228, 228, 228, 228, 228, 20,  20,  
	20,  20,  20,  228, 228, 228, 228, 228, 228, 228, 228, 228, 228, 20,  20,  20,  20,  20,  20,  228, 228, 228, 228, 228, 228, 228, 228, 228, 228, 228, 228, 228, 228, 228, 228, 228, 228, 228, 228, 228, 228, 228, 228, 228, 228, 152, 152, 152, 152, 152, 152, 228, 228, 228, 228, 228, 228, 228, 228, 228, 228, 152, 152, 152, 152, 152, 152, 228, 228, 228, 228, 228, 228, 228, 228, 228, 228, 228, 228, 228, 228, 228, 228, 228, 228, 228, 228, 228, 228, 228, 228, 228, 228, 173, 173, 173, 173, 173, 173, 228, 228, 228, 228, 228, 228, 228, 228, 228, 228, 173, 173, 173, 173, 173, 173, 228, 228, 228, 228, 228, 228, 228, 228, 228, 228, 228, 228, 228, 228, 228, 228, 228, 228, 228, 228, 228, 228, 228, 228, 228, 228, 228, 228, 228, 255, 228, 228, 228, 228, 228, 228, 228, 228, 228, 228, 228, 228, 20,  20,  20,  
	20,  20,  20,  20,  228, 228, 228, 228, 228, 228, 228, 228, 20,  20,  20,  20,  20,  20,  20,  20,  228, 228, 228, 228, 228, 228, 228, 228, 228, 228, 228, 228, 228, 228, 228, 228, 228, 228, 228, 228, 228, 228, 228, 228, 152, 152, 152, 152, 152, 152, 152, 152, 228, 228, 228, 228, 228, 228, 228, 228, 152, 152, 152, 152, 152, 152, 152, 152, 228, 228, 228, 228, 228, 228, 228, 228, 228, 228, 228, 228, 228, 228, 228, 228, 228, 228, 228, 228, 228, 228, 228, 228, 173, 173, 173, 173, 173, 173, 173, 173, 228, 228, 228, 228, 228, 228, 228, 228, 173, 173, 173, 173, 173, 173, 173, 173, 228, 228, 228, 228, 228, 228, 228, 228, 228, 228, 228, 228, 228, 228, 228, 228, 228, 228, 228, 228, 228, 228, 228, 228, 228, 228, 228, 228, 255, 228, 228, 228, 228, 228, 228, 228, 228, 228, 228, 228, 20,  20,  20,  20,  
	20,  20,  20,  20,  20,  228, 228, 228, 228, 228, 228, 20,  20,  20,  20,  20,  20,  20,  20,  20,  20,  228, 228, 228, 228, 228, 228, 228, 228, 228, 228, 228, 228, 228, 228, 228, 228, 228, 228, 228, 228, 228, 228, 152, 152, 152, 152, 152, 152, 152, 152, 152, 152, 228, 228, 228, 228, 228, 228, 152, 152, 152, 152, 152, 152, 152, 152, 152, 152, 228, 228, 228, 228, 228, 228, 228, 228, 228, 228, 228, 228, 228, 228, 228, 228, 228, 228, 228, 228, 228, 228, 173, 173, 173, 173, 173, 173, 173, 173, 173, 173, 228, 228, 228, 228, 228, 228, 173, 173, 173, 173, 173, 173, 173, 173, 173, 173, 228, 228, 228, 228, 228, 228, 228, 228, 228, 228, 228, 228, 228, 228, 228, 228, 228, 228, 228, 228, 228, 228, 228, 228, 228, 228, 255, 38,  38,  228, 228, 228, 228, 228, 228, 228, 228, 228, 20,  20,  20,  20,  20,  
	20,  20,  20,  0,   0,   20,  228, 228, 228, 228, 20,  0,   0,   20,  20,  20,  20,  20,  20,  0,   0,   20,  228, 228, 228, 228, 228, 228, 228, 228, 228, 228, 228, 228, 228, 228, 228, 228, 228, 228, 228, 228, 152, 80,  80,  152, 152, 152, 152, 152, 152, 80,  80,  152, 228, 228, 228, 228, 152, 80,  80,  152, 152, 152, 152, 152, 152, 80,  80,  152, 228, 228, 228, 228, 228, 228, 228, 228, 228, 228, 228, 228, 228, 228, 228, 228, 228, 228, 228, 228, 173, 0,   0,   173, 173, 173, 173, 173, 173, 0,   0,   173, 228, 228, 228, 228, 173, 0,   0,   173, 173, 173, 173, 173, 173, 0,   0,   173, 228, 228, 228, 228, 228, 228, 228, 228, 228, 228, 228, 228, 228, 228, 228, 228, 228, 228, 228, 228, 255, 228, 228, 228, 228, 255, 38,  38,  228, 228, 228, 228, 255, 228, 228, 228, 20,  0,   0,   20,  20,  20,  
	20,  20,  0,   183, 20,  20,  20,  228, 228, 20,  20,  20,  183, 0,   20,  20,  20,  20,  0,   183, 20,  20,  20,  228, 228, 228, 228, 228, 228, 228, 228, 228, 228, 228, 228, 228, 228, 228, 228, 228, 228, 152, 152, 152, 253, 80,  152, 152, 152, 152, 80,  253, 152, 152, 152, 228, 228, 152, 152, 152, 253, 80,  152, 152, 152, 152, 80,  253, 152, 152, 152, 228, 228, 228, 228, 228, 228, 228, 228, 228, 228, 228, 228, 228, 228, 228, 228, 228, 228, 173, 173, 173, 255, 0,   173, 173, 173, 173, 0,   255, 173, 173, 173, 228, 228, 173, 173, 173, 255, 0,   173, 173, 173, 173, 0,   255, 173, 173, 173, 228, 228, 228, 228, 228, 228, 228, 228, 228, 228, 228, 228, 228, 228, 228, 228, 228, 228, 228, 255, 38,  228, 228, 255, 38,  38,  38,  38,  228, 228, 255, 38,  228, 228, 20,  20,  20,  183, 0,   20,  20,  
	0,   0,   0,   183, 20,  20,  20,  228, 228, 20,  20,  20,  183, 0,   0,   0,   0,   0,   0,   183, 20,  20,  20,  228, 228, 228, 228, 228, 228, 228, 228, 228, 228, 228, 228, 228, 228, 228, 228, 228, 228, 152, 152, 152, 253, 80,  80,  80,  80,  80,  80,  253, 152, 152, 152, 228, 228, 152, 152, 152, 253, 80,  80,  80,  80,  80,  80,  253, 152, 152, 152, 228, 228, 228, 228, 228, 228, 228, 228, 228, 228, 228, 228, 228, 228, 228, 228, 228, 228, 173, 173, 173, 255, 0,   0,   0,   0,   0,   0,   255, 173, 173, 173, 228, 228, 173, 173, 173, 255, 0,   0,   0,   0,   0,   0,   255, 173, 173, 173, 228, 228, 228, 228, 228, 228, 228, 228, 228, 228, 228, 228, 228, 228, 228, 228, 228, 228, 228, 255, 38,  38,  228, 255, 38,  38,  38,  38,  228, 255, 38,  38,  228, 228, 20,  20,  20,  183, 0,   0,   0,   
	20,  183, 0,   183, 20,  20,  20,  20,  20,  20,  20,  20,  183, 0,   183, 20,  20,  183, 0,   183, 20,  20,  20,  20,  228, 228, 228, 228, 228, 228, 228, 228, 228, 228, 228, 228, 228, 228, 228, 228, 152, 152, 152, 152, 253, 80,  253, 152, 152, 253, 80,  253, 152, 152, 152, 152, 152, 152, 152, 152, 253, 80,  253, 152, 152, 253, 80,  253, 152, 152, 152, 152, 228, 228, 228, 228, 228, 228, 228, 228, 228, 228, 228, 228, 228, 228, 228, 228, 173, 173, 173, 173, 255, 0,   255, 173, 173, 255, 0,   255, 173, 173, 173, 173, 173, 173, 173, 173, 255, 0,   255, 173, 173, 255, 0,   255, 173, 173, 173, 173, 228, 228, 228, 228, 228, 228, 228, 228, 228, 228, 228, 228, 228, 228, 228, 228, 228, 228, 255, 38,  38,  38,  13,  38,  38,  38,  13,  255, 38,  38,  38,  228, 20,  20,  20,  20,  183, 0,   183, 20,  
	20,  183, 183, 183, 20,  20,  20,  20,  20,  20,  20,  20,  183, 183, 183, 20,  20,  183, 183, 183, 20,  20,  20,  20,  228, 228, 228, 228, 228, 228, 20,  20,  20,  20,  228, 228, 228, 228, 228, 228, 152, 152, 152, 152, 253, 253, 253, 152, 152, 253, 253, 253, 152, 152, 152, 152, 152, 152, 152, 152, 253, 253, 253, 152, 152, 253, 253, 253, 152, 152, 152, 152, 228, 228, 228, 228, 228, 228, 152, 152, 152, 152, 228, 228, 228, 228, 228, 228, 173, 173, 173, 173, 255, 255, 255, 173, 173, 255, 255, 255, 173, 173, 173, 173, 173, 173, 173, 173, 255, 255, 255, 173, 173, 255, 255, 255, 173, 173, 173, 173, 228, 228, 228, 228, 228, 228, 173, 173, 173, 173, 228, 228, 228, 228, 228, 228, 228, 228, 255, 255, 38,  38,  13,  13,  13,  13,  13,  255, 38,  38,  38,  228, 20,  20,  20,  20,  183, 183, 183, 20,  
	20,  20,  20,  20,  20,  20,  20,  20,  20,  20,  20,  20,  20,  20,  20,  20,  20,  20,  20,  20,  20,  20,  20,  20,  228, 228, 228, 20,  20,  20,  20,  20,  20,  20,  20,  20,  20,  228, 228, 228, 152, 152, 152, 152, 152, 152, 152, 152, 152, 152, 152, 152, 152, 152, 152, 152, 152, 152, 152, 152, 152, 152, 152, 152, 152, 152, 152, 152, 152, 152, 152, 152, 228, 228, 228, 152, 152, 152, 152, 152, 152, 152, 152, 152, 152, 228, 228, 228, 173, 173, 173, 173, 173, 173, 173, 173, 173, 173, 173, 173, 173, 173, 173, 173, 173, 173, 173, 173, 173, 173, 173, 173, 173, 173, 173, 173, 173, 173, 173, 173, 228, 228, 228, 173, 173, 173, 173, 173, 173, 173, 173, 173, 173, 228, 228, 228, 228, 13,  13,  38,  38,  13,  13,  255, 255, 38,  13,  13,  38,  38,  13,  228, 20,  20,  20,  20,  20,  20,  20,  20,  
	183, 183, 183, 20,  20,  20,  20,  228, 228, 20,  20,  20,  20,  183, 183, 183, 183, 183, 183, 20,  20,  20,  20,  228, 228, 20,  20,  0,   0,   0,   20,  20,  20,  20,  0,   0,   0,   20,  20,  228, 228, 152, 152, 152, 152, 253, 253, 253, 253, 253, 253, 152, 152, 152, 152, 228, 228, 152, 152, 152, 152, 253, 253, 253, 253, 253, 253, 152, 152, 152, 152, 228, 228, 152, 152, 80,  80,  80,  152, 152, 152, 152, 80,  80,  80,  152, 152, 228, 228, 173, 173, 173, 173, 255, 255, 255, 255, 255, 255, 173, 173, 173, 173, 228, 228, 173, 173, 173, 173, 255, 255, 255, 255, 255, 255, 173, 173, 173, 173, 228, 228, 173, 173, 0,   0,   0,   173, 173, 173, 173, 0,   0,   0,   173, 173, 228, 255, 255, 255, 13,  13,  13,  255, 255, 38,  38,  38,  13,  13,  13,  13,  228, 228, 20,  20,  20,  20,  183, 183, 183, 
	183, 183, 183, 183, 228, 228, 228, 228, 228, 228, 228, 228, 183, 183, 183, 183, 183, 183, 183, 183, 228, 228, 228, 228, 20,  20,  183, 183, 183, 183, 0,   0,   0,   0,   183, 183, 183, 183, 20,  20,  228, 228, 228, 228, 253, 253, 253, 253, 253, 253, 253, 253, 228, 228, 228, 228, 228, 228, 228, 228, 253, 253, 253, 253, 253, 253, 253, 253, 228, 228, 228, 228, 152, 152, 253, 253, 253, 253, 80,  80,  80,  80,  253, 253, 253, 253, 152, 152, 228, 228, 228, 228, 255, 255, 255, 255, 255, 255, 255, 255, 228, 228, 228, 228, 228, 228, 228, 228, 255, 255, 255, 255, 255, 255, 255, 255, 228, 228, 228, 228, 173, 173, 255, 255, 255, 255, 0,   0,   0,   0,   255, 255, 255, 255, 173, 173, 228, 13,  13,  255, 13,  13,  38,  38,  38,  38,  38,  13,  13,  13,  13,  13,  228, 228, 228, 228, 183, 183, 183, 183, 
	183, 183, 183, 183, 0,   0,   228, 228, 228, 228, 0,   0,   183, 183, 183, 183, 183, 183, 183, 183, 228, 228, 228, 228, 20,  20,  20,  20,  20,  20,  20,  20,  20,  20,  20,  20,  20,  20,  20,  20,  228, 228, 228, 228, 253, 253, 253, 253, 253, 253, 253, 253, 80,  80,  228, 228, 228, 228, 80,  80,  253, 253, 253, 253, 253, 253, 253, 253, 228, 228, 228, 228, 152, 152, 152, 152, 152, 152, 152, 152, 152, 152, 152, 152, 152, 152, 152, 152, 228, 228, 228, 228, 255, 255, 255, 255, 255, 255, 255, 255, 0,   0,   228, 228, 228, 228, 0,   0,   255, 255, 255, 255, 255, 255, 255, 255, 228, 228, 228, 228, 173, 173, 173, 173, 173, 173, 173, 173, 173, 173, 173, 173, 173, 173, 173, 173, 13,  255, 13,  13,  255, 13,  13,  38,  38,  38,  13,  13,  13,  255, 255, 255, 228, 228, 228, 228, 183, 183, 183, 183, 
	183, 183, 0,   0,   0,   0,   0,   228, 228, 0,   0,   0,   0,   0,   183, 183, 183, 183, 183, 0,   0,   228, 228, 228, 228, 228, 228, 183, 183, 183, 183, 183, 183, 183, 183, 183, 183, 228, 228, 228, 228, 228, 228, 80,  80,  253, 253, 253, 253, 253, 80,  80,  80,  80,  80,  228, 228, 80,  80,  80,  80,  80,  253, 253, 253, 253, 253, 80,  80,  228, 228, 228, 228, 228, 228, 253, 253, 253, 253, 253, 253, 253, 253, 253, 253, 228, 228, 228, 228, 228, 228, 0,   0,   255, 255, 255, 255, 255, 0,   0,   0,   0,   0,   228, 228, 0,   0,   0,   0,   0,   255, 255, 255, 255, 255, 0,   0,   228, 228, 228, 228, 228, 228, 255, 255, 255, 255, 255, 255, 255, 255, 255, 255, 228, 228, 228, 13,  13,  13,  13,  38,  255, 13,  13,  13,  13,  13,  255, 255, 255, 228, 228, 228, 228, 228, 0,   0,   183, 183, 183, 
	183, 0,   0,   0,   0,   0,   0,   228, 228, 0,   0,   0,   0,   0,   0,   183, 183, 183, 0,   0,   0,   228, 228, 228, 228, 228, 228, 228, 183, 183, 183, 183, 183, 183, 183, 183, 228, 228, 228, 228, 228, 228, 228, 80,  80,  80,  253, 253, 253, 80,  80,  80,  80,  80,  80,  228, 228, 80,  80,  80,  80,  80,  80,  253, 253, 253, 80,  80,  80,  228, 228, 228, 228, 228, 228, 228, 253, 253, 253, 253, 253, 253, 253, 253, 228, 228, 228, 228, 228, 228, 228, 0,   0,   0,   255, 255, 255, 0,   0,   0,   0,   0,   0,   228, 228, 0,   0,   0,   0,   0,   0,   255, 255, 255, 0,   0,   0,   228, 228, 228, 228, 228, 228, 228, 255, 255, 255, 255, 255, 255, 255, 255, 228, 228, 228, 228, 228, 228, 228, 38,  38,  38,  255, 255, 255, 255, 255, 255, 38,  38,  38,  228, 228, 228, 228, 0,   0,   0,   183, 183, 
	228, 0,   0,   0,   0,   0,   228, 228, 228, 228, 0,   0,   0,   0,   0,   228, 228, 0,   0,   0,   228, 228, 228, 228, 228, 0,   0,   0,   0,   0,   228, 228, 228, 228, 0,   0,   0,   0,   0,   228, 228, 228, 228, 228, 80,  80,  80,  228, 228, 80,  80,  80,  80,  80,  228, 228, 228, 228, 80,  80,  80,  80,  80,  228, 228, 80,  80,  80,  228, 228, 228, 228, 228, 80,  80,  80,  80,  80,  228, 228, 228, 228, 80,  80,  80,  80,  80,  228, 228, 228, 228, 228, 0,   0,   0,   228, 228, 0,   0,   0,   0,   0,   228, 228, 228, 228, 0,   0,   0,   0,   0,   228, 228, 0,   0,   0,   228, 228, 228, 228, 228, 0,   0,   0,   0,   0,   228, 228, 228, 228, 0,   0,   0,   0,   0,   228, 228, 228, 38,  38,  38,  38,  228, 228, 228, 228, 228, 228, 38,  38,  38,  38,  228, 228, 228, 228, 0,   0,   0,   228   
}

\def\arrayHalfEnemiesTwo{
	228, 228, 228, 228, 228, 228, 228, 228, 228, 228, 228, 228, 228, 228, 228, 0,   0,   0,   228, 228, 228, 228, 228, 228, 228, 228, 228, 228, 228, 228, 228, 228, 228, 228, 228, 228, 228, 228, 228, 228, 228, 228, 228, 228, 228, 228, 228, 228, 228, 228, 228, 228, 228, 228, 228, 228, 228, 228, 228, 228, 228, 228, 228, 80,  80,  80,  228, 228, 228, 228, 228, 228, 228, 228, 228, 228, 228, 228, 228, 228, 228, 228, 228, 228, 228, 228, 228, 228, 228, 228, 228, 228, 228, 228, 228, 228, 228, 228, 228, 228, 228, 228, 228, 228, 228, 228, 228, 228, 228, 228, 228, 0,   0,   0,   228, 228, 228, 228, 228, 228, 228, 228, 228, 228, 228, 228, 228, 228, 228, 228, 228, 228, 228, 228, 228, 228, 228, 228, 228, 228, 228, 228, 228, 38,  38,  228, 228, 228, 228, 228, 228, 228, 228, 228, 228, 228, 228, 228, 228, 228, 
	0,   0,   228, 228, 228, 228, 228, 228, 228, 228, 228, 228, 0,   0,   0,   0,   0,   0,   0,   0,   228, 228, 228, 228, 228, 228, 228, 228, 228, 228, 0,   0,   0,   0,   228, 228, 228, 228, 228, 228, 228, 228, 228, 228, 228, 228, 80,  80,  80,  80,  228, 228, 228, 228, 228, 228, 228, 228, 228, 228, 80,  80,  80,  80,  80,  80,  80,  80,  228, 228, 228, 228, 228, 228, 228, 228, 228, 228, 80,  80,  80,  80,  228, 228, 228, 228, 228, 228, 228, 228, 228, 228, 228, 228, 0,   0,   0,   0,   228, 228, 228, 228, 228, 228, 228, 228, 228, 228, 0,   0,   0,   0,   0,   0,   0,   0,   228, 228, 228, 228, 228, 228, 228, 228, 228, 228, 0,   0,   0,   0,   228, 228, 228, 228, 228, 228, 228, 228, 228, 228, 38,  38,  13,  38,  38,  13,  38,  38,  228, 228, 228, 228, 228, 228, 228, 228, 228, 228, 0,   0,   
	0,   0,   0,   0,   228, 228, 228, 228, 228, 228, 228, 0,   0,   0,   0,   0,   0,   0,   0,   0,   0,   228, 228, 228, 228, 228, 228, 228, 0,   0,   0,   0,   0,   0,   0,   0,   228, 228, 228, 228, 228, 228, 228, 228, 80,  80,  80,  80,  80,  80,  80,  80,  228, 228, 228, 228, 228, 228, 228, 80,  80,  80,  80,  80,  80,  80,  80,  80,  80,  228, 228, 228, 228, 228, 228, 228, 80,  80,  80,  80,  80,  80,  80,  80,  228, 228, 228, 228, 228, 228, 228, 228, 0,   0,   0,   0,   0,   0,   0,   0,   228, 228, 228, 228, 228, 228, 228, 0,   0,   0,   0,   0,   0,   0,   0,   0,   0,   228, 228, 228, 228, 228, 228, 228, 0,   0,   0,   0,   0,   0,   0,   0,   228, 228, 228, 228, 228, 228, 228, 228, 38,  13,  13,  13,  13,  13,  13,  38,  228, 228, 228, 228, 228, 228, 228, 228, 0,   0,   0,   0,   
	0,   0,   0,   0,   0,   228, 228, 228, 228, 228, 0,   0,   0,   0,   0,   0,   0,   20,  20,  0,   0,   0,   228, 228, 228, 228, 228, 0,   0,   0,   0,   20,  20,  0,   0,   0,   0,   228, 228, 228, 228, 228, 228, 80,  80,  80,  80,  80,  80,  80,  80,  80,  80,  228, 228, 228, 228, 228, 80,  80,  80,  80,  80,  80,  80,  152, 152, 80,  80,  80,  228, 228, 228, 228, 228, 80,  80,  80,  80,  152, 152, 80,  80,  80,  80,  228, 228, 228, 228, 228, 228, 0,   0,   0,   0,   0,   0,   0,   0,   0,   0,   228, 228, 228, 228, 228, 0,   0,   0,   0,   0,   0,   0,   173, 173, 0,   0,   0,   228, 228, 228, 228, 228, 0,   0,   0,   0,   173, 173, 0,   0,   0,   0,   228, 228, 228, 228, 228, 228, 13,  13,  13,  13,  13,  13,  13,  13,  13,  13,  228, 228, 228, 228, 228, 228, 0,   0,   0,   0,   0,   
	0,   20,  20,  0,   0,   0,   228, 228, 228, 0,   0,   0,   0,   0,   0,   0,   0,   20,  183, 20,  0,   0,   228, 228, 228, 228, 0,   0,   0,   0,   20,  183, 183, 20,  0,   0,   0,   0,   228, 228, 228, 228, 80,  80,  80,  80,  80,  80,  80,  152, 152, 80,  80,  80,  228, 228, 228, 80,  80,  80,  80,  80,  80,  80,  80,  152, 253, 152, 80,  80,  228, 228, 228, 228, 80,  80,  80,  80,  152, 253, 253, 152, 80,  80,  80,  80,  228, 228, 228, 228, 0,   0,   0,   0,   0,   0,   0,   173, 173, 0,   0,   0,   228, 228, 228, 0,   0,   0,   0,   0,   0,   0,   0,   173, 255, 173, 0,   0,   228, 228, 228, 228, 0,   0,   0,   0,   173, 255, 255, 173, 0,   0,   0,   0,   228, 228, 228, 228, 38,  38,  13,  255, 255, 13,  13,  13,  13,  13,  38,  38,  228, 228, 228, 228, 0,   0,   0,   0,   0,   0,   
	0,   20,  183, 20,  0,   0,   228, 228, 228, 0,   0,   0,   0,   0,   0,   0,   0,   0,   20,  20,  0,   0,   228, 228, 228, 228, 0,   0,   0,   0,   0,   20,  20,  0,   0,   0,   0,   0,   228, 228, 228, 80,  80,  80,  80,  80,  80,  80,  80,  152, 253, 152, 80,  80,  228, 228, 228, 80,  80,  80,  80,  80,  80,  80,  80,  80,  152, 152, 80,  80,  228, 228, 228, 228, 80,  80,  80,  80,  80,  152, 152, 80,  80,  80,  80,  80,  228, 228, 228, 0,   0,   0,   0,   0,   0,   0,   0,   173, 255, 173, 0,   0,   228, 228, 228, 0,   0,   0,   0,   0,   0,   0,   0,   0,   173, 173, 0,   0,   228, 228, 228, 228, 0,   0,   0,   0,   0,   173, 173, 0,   0,   0,   0,   0,   228, 228, 228, 228, 38,  13,  255, 255, 13,  13,  13,  13,  13,  13,  13,  38,  228, 228, 228, 0,   0,   0,   0,   0,   0,   0,   
	0,   0,   20,  20,  0,   0,   228, 228, 20,  20,  20,  20,  0,   0,   0,   0,   0,   0,   0,   0,   0,   0,   0,   228, 228, 228, 0,   0,   0,   0,   0,   0,   0,   0,   0,   0,   0,   0,   228, 228, 228, 80,  80,  80,  80,  80,  80,  80,  80,  80,  152, 152, 80,  80,  228, 228, 152, 152, 152, 152, 80,  80,  80,  80,  80,  80,  80,  80,  80,  80,  80,  228, 228, 228, 80,  80,  80,  80,  80,  80,  80,  80,  80,  80,  80,  80,  228, 228, 228, 0,   0,   0,   0,   0,   0,   0,   0,   0,   173, 173, 0,   0,   228, 228, 173, 173, 173, 173, 0,   0,   0,   0,   0,   0,   0,   0,   0,   0,   0,   228, 228, 228, 0,   0,   0,   0,   0,   0,   0,   0,   0,   0,   0,   0,   228, 228, 228, 228, 13,  13,  255, 13,  13,  13,  13,  13,  13,  13,  13,  13,  228, 228, 228, 0,   0,   0,   0,   0,   0,   0,   
	0,   0,   0,   0,   0,   0,   0,   228, 228, 0,   0,   20,  20,  0,   0,   0,   0,   0,   0,   0,   0,   0,   0,   228, 228, 0,   0,   0,   0,   0,   0,   0,   0,   0,   0,   0,   0,   0,   0,   228, 152, 152, 152, 152, 80,  80,  80,  80,  80,  80,  80,  80,  80,  80,  80,  228, 228, 80,  80,  152, 152, 80,  80,  80,  80,  80,  80,  80,  80,  80,  80,  228, 228, 80,  80,  80,  80,  80,  80,  80,  80,  80,  80,  80,  80,  80,  80,  228, 173, 173, 173, 173, 0,   0,   0,   0,   0,   0,   0,   0,   0,   0,   0,   228, 228, 0,   0,   173, 173, 0,   0,   0,   0,   0,   0,   0,   0,   0,   0,   228, 228, 0,   0,   0,   0,   0,   0,   0,   0,   0,   0,   0,   0,   0,   0,   228, 228, 38,  38,  13,  13,  13,  13,  13,  13,  13,  13,  13,  13,  38,  38,  228, 20,  20,  20,  20,  0,   0,   0,   0,   
	0,   0,   0,   0,   0,   0,   0,   228, 228, 0,   0,   0,   20,  0,   0,   0,   0,   0,   0,   0,   0,   0,   0,   228, 228, 0,   0,   0,   0,   0,   0,   0,   0,   0,   0,   0,   0,   0,   0,   228, 228, 80,  80,  152, 152, 80,  80,  80,  80,  80,  80,  80,  80,  80,  80,  228, 228, 80,  80,  80,  152, 80,  80,  80,  80,  80,  80,  80,  80,  80,  80,  228, 228, 80,  80,  80,  80,  80,  80,  80,  80,  80,  80,  80,  80,  80,  80,  228, 228, 0,   0,   173, 173, 0,   0,   0,   0,   0,   0,   0,   0,   0,   0,   228, 228, 0,   0,   0,   173, 0,   0,   0,   0,   0,   0,   0,   0,   0,   0,   228, 228, 0,   0,   0,   0,   0,   0,   0,   0,   0,   0,   0,   0,   0,   0,   228, 228, 38,  38,  13,  13,  13,  13,  13,  13,  13,  13,  13,  13,  38,  38,  228, 228, 0,   0,   20,  20,  0,   0,   0,   
	0,   0,   0,   0,   0,   0,   0,   228, 0,   0,   183, 0,   20,  0,   0,   0,   0,   0,   0,   0,   0,   0,   0,   228, 228, 0,   0,   0,   0,   0,   0,   0,   0,   0,   0,   0,   0,   0,   0,   228, 228, 80,  80,  80,  152, 80,  80,  80,  80,  80,  80,  80,  80,  80,  80,  228, 80,  80,  253, 80,  152, 80,  80,  80,  80,  80,  80,  80,  80,  80,  80,  228, 228, 80,  80,  80,  80,  80,  80,  80,  80,  80,  80,  80,  80,  80,  80,  228, 228, 0,   0,   0,   173, 0,   0,   0,   0,   0,   0,   0,   0,   0,   0,   228, 0,   0,   255, 0,   173, 0,   0,   0,   0,   0,   0,   0,   0,   0,   0,   228, 228, 0,   0,   0,   0,   0,   0,   0,   0,   0,   0,   0,   0,   0,   0,   228, 228, 228, 13,  13,  13,  13,  13,  13,  13,  13,  13,  255, 13,  13,  228, 228, 228, 0,   0,   0,   20,  0,   0,   0,   
	0,   0,   0,   0,   0,   0,   0,   228, 0,   0,   0,   0,   20,  0,   0,   0,   0,   0,   0,   0,   0,   0,   0,   228, 228, 0,   0,   0,   0,   0,   0,   0,   0,   0,   0,   0,   0,   0,   0,   228, 80,  80,  253, 80,  152, 80,  80,  80,  80,  80,  80,  80,  80,  80,  80,  228, 80,  80,  80,  80,  152, 80,  80,  80,  80,  80,  80,  80,  80,  80,  80,  228, 228, 80,  80,  80,  80,  80,  80,  80,  80,  80,  80,  80,  80,  80,  80,  228, 0,   0,   255, 0,   173, 0,   0,   0,   0,   0,   0,   0,   0,   0,   0,   228, 0,   0,   0,   0,   173, 0,   0,   0,   0,   0,   0,   0,   0,   0,   0,   228, 228, 0,   0,   0,   0,   0,   0,   0,   0,   0,   0,   0,   0,   0,   0,   228, 228, 228, 38,  13,  13,  13,  13,  13,  13,  13,  255, 255, 13,  38,  228, 228, 0,   0,   183, 0,   20,  0,   0,   0,   
	0,   0,   0,   0,   0,   0,   0,   228, 0,   0,   0,   0,   20,  0,   0,   0,   0,   0,   0,   0,   0,   0,   0,   228, 228, 0,   0,   0,   0,   0,   0,   0,   0,   0,   0,   0,   0,   0,   0,   228, 80,  80,  80,  80,  152, 80,  80,  80,  80,  80,  80,  80,  80,  80,  80,  228, 80,  80,  80,  80,  152, 80,  80,  80,  80,  80,  80,  80,  80,  80,  80,  228, 228, 80,  80,  80,  80,  80,  80,  80,  80,  80,  80,  80,  80,  80,  80,  228, 0,   0,   0,   0,   173, 0,   0,   0,   0,   0,   0,   0,   0,   0,   0,   228, 0,   0,   0,   0,   173, 0,   0,   0,   0,   0,   0,   0,   0,   0,   0,   228, 228, 0,   0,   0,   0,   0,   0,   0,   0,   0,   0,   0,   0,   0,   0,   228, 228, 228, 38,  38,  13,  13,  13,  13,  13,  255, 255, 13,  38,  38,  228, 228, 0,   0,   0,   0,   20,  0,   0,   0,   
	0,   0,   0,   0,   0,   0,   0,   228, 228, 0,   0,   183, 20,  20,  0,   0,   0,   0,   0,   20,  20,  20,  20,  20,  228, 0,   0,   0,   0,   0,   0,   0,   0,   0,   0,   0,   0,   0,   0,   228, 80,  80,  80,  80,  152, 80,  80,  80,  80,  80,  80,  80,  80,  80,  80,  228, 228, 80,  80,  253, 152, 152, 80,  80,  80,  80,  80,  152, 152, 152, 152, 152, 228, 80,  80,  80,  80,  80,  80,  80,  80,  80,  80,  80,  80,  80,  80,  228, 0,   0,   0,   0,   173, 0,   0,   0,   0,   0,   0,   0,   0,   0,   0,   228, 228, 0,   0,   255, 173, 173, 0,   0,   0,   0,   0,   173, 173, 173, 173, 173, 228, 0,   0,   0,   0,   0,   0,   0,   0,   0,   0,   0,   0,   0,   0,   228, 228, 228, 228, 13,  13,  13,  13,  13,  13,  13,  13,  13,  13,  228, 228, 228, 0,   0,   0,   0,   20,  0,   0,   0,   
	0,   0,   0,   20,  20,  20,  20,  20,  228, 228, 228, 183, 183, 20,  20,  0,   0,   20,  20,  20,  183, 183, 228, 228, 20,  20,  20,  20,  20,  0,   0,   0,   0,   0,   0,   20,  20,  20,  20,  20,  228, 80,  80,  253, 152, 152, 80,  80,  80,  80,  80,  152, 152, 152, 152, 152, 228, 228, 228, 253, 253, 152, 152, 80,  80,  152, 152, 152, 253, 253, 228, 228, 152, 152, 152, 152, 152, 80,  80,  80,  80,  80,  80,  152, 152, 152, 152, 152, 228, 0,   0,   255, 173, 173, 0,   0,   0,   0,   0,   173, 173, 173, 173, 173, 228, 228, 228, 255, 255, 173, 173, 0,   0,   173, 173, 173, 255, 255, 228, 228, 173, 173, 173, 173, 173, 0,   0,   0,   0,   0,   0,   173, 173, 173, 173, 173, 228, 228, 228, 228, 38,  13,  13,  13,  13,  13,  13,  38,  228, 228, 228, 228, 228, 0,   0,   183, 20,  20,  0,   0,   
	0,   20,  20,  20,  183, 183, 183, 228, 228, 228, 228, 183, 183, 183, 20,  20,  20,  20,  228, 183, 183, 183, 228, 228, 228, 228, 228, 228, 20,  20,  20,  0,   0,   20,  20,  20,  228, 228, 228, 228, 228, 228, 253, 253, 253, 152, 152, 80,  80,  152, 152, 152, 253, 253, 253, 228, 228, 228, 228, 253, 253, 253, 152, 152, 152, 152, 228, 253, 253, 253, 228, 228, 228, 228, 228, 228, 152, 152, 152, 80,  80,  152, 152, 152, 228, 228, 228, 228, 228, 228, 255, 255, 255, 173, 173, 0,   0,   173, 173, 173, 255, 255, 255, 228, 228, 228, 228, 255, 255, 255, 173, 173, 173, 173, 228, 255, 255, 255, 228, 228, 228, 228, 228, 228, 173, 173, 173, 0,   0,   173, 173, 173, 228, 228, 228, 228, 228, 228, 228, 228, 38,  38,  13,  38,  38,  13,  38,  38,  228, 228, 228, 228, 228, 228, 183, 183, 183, 20,  20,  0,   
	20,  20,  228, 228, 183, 183, 183, 183, 228, 228, 228, 228, 183, 183, 228, 228, 228, 228, 228, 183, 183, 228, 228, 228, 228, 228, 228, 228, 228, 228, 20,  20,  20,  20,  228, 228, 228, 228, 228, 228, 228, 253, 253, 253, 253, 228, 152, 152, 152, 152, 228, 228, 253, 253, 253, 253, 228, 228, 228, 228, 253, 253, 228, 228, 228, 228, 228, 253, 253, 228, 228, 228, 228, 228, 228, 228, 228, 228, 152, 152, 152, 152, 228, 228, 228, 228, 228, 228, 228, 136, 255, 255, 255, 228, 173, 173, 173, 173, 228, 228, 255, 255, 255, 255, 228, 228, 228, 228, 255, 255, 228, 228, 228, 228, 228, 255, 255, 228, 228, 228, 228, 228, 228, 228, 228, 228, 173, 173, 173, 173, 228, 228, 228, 228, 228, 228, 228, 228, 228, 228, 228, 228, 228, 38,  38,  228, 228, 228, 228, 228, 228, 228, 228, 183, 183, 183, 183, 228, 20,  20 
}

\def\arrayHalfEnemiesThree{
	228, 228, 228, 228, 228, 228, 228, 228, 228, 228, 228, 228, 228, 228, 228, 228, 228, 228, 228, 228, 228, 228, 228, 228, 228, 228, 228, 228, 228, 38,  38,  38,  38,  38,  228, 228, 228, 228, 228, 228, 228, 228, 228, 228, 228, 228, 228, 228, 228, 228, 228, 228, 183, 183, 183, 228, 228, 228, 228, 228, 228, 228, 228, 228, 228, 228, 228, 228, 228, 228, 228, 228, 228, 228, 228, 228, 228, 20,  20,  20,  20,  20,  228, 228, 228, 228, 228, 228, 228, 228, 228, 228, 228, 228, 228, 228, 228, 228, 228, 228, 255, 255, 255, 228, 228, 228, 228, 228, 228, 228, 228, 228, 228, 228, 228, 228, 228, 228, 228, 228, 228, 228, 228, 228, 228, 38,  38,  38,  38,  38,  228, 228, 228, 228, 228, 228, 228, 228, 228, 228, 228, 228, 228, 228, 228, 228, 228, 228, 228, 228, 228, 228, 228, 228, 228, 228, 228, 228, 228, 228, 
	32,  32,  228, 228, 228, 38,  228, 228, 228, 228, 228, 228, 228, 228, 228, 228, 228, 228, 228, 228, 228, 228, 228, 228, 228, 228, 228, 228, 38,  38,  38,  38,  38,  38,  38,  228, 228, 228, 228, 228, 228, 228, 228, 228, 228, 228, 228, 228, 152, 152, 183, 183, 183, 20,  228, 228, 228, 228, 228, 228, 228, 228, 228, 228, 228, 228, 228, 228, 228, 228, 228, 228, 228, 228, 228, 228, 20,  20,  20,  20,  20,  20,  20,  228, 228, 228, 228, 228, 228, 228, 228, 228, 228, 228, 228, 228, 173, 173, 255, 255, 255, 38,  228, 228, 228, 228, 228, 228, 228, 228, 228, 228, 228, 228, 228, 228, 228, 228, 228, 228, 228, 228, 228, 228, 38,  38,  38,  38,  38,  38,  38,  228, 228, 228, 228, 228, 228, 228, 228, 228, 228, 228, 228, 228, 228, 228, 228, 228, 228, 228, 228, 228, 228, 228, 228, 228, 228, 228, 228, 228, 
	32,  32,  228, 228, 38,  38,  228, 228, 228, 228, 228, 228, 228, 228, 228, 228, 228, 228, 228, 228, 228, 228, 228, 228, 228, 228, 228, 32,  32,  32,  38,  32,  32,  32,  38,  38,  228, 228, 228, 228, 228, 228, 228, 228, 228, 228, 228, 152, 152, 152, 183, 183, 20,  20,  228, 228, 228, 228, 228, 228, 228, 228, 228, 228, 228, 228, 228, 228, 228, 228, 228, 228, 228, 228, 228, 152, 152, 152, 20,  152, 152, 152, 20,  20,  228, 228, 228, 228, 228, 228, 228, 228, 228, 228, 228, 173, 173, 173, 255, 255, 38,  38,  228, 228, 228, 228, 228, 228, 228, 228, 228, 228, 228, 228, 228, 228, 228, 228, 228, 228, 228, 228, 228, 173, 173, 173, 38,  173, 173, 173, 38,  38,  228, 228, 228, 228, 228, 228, 228, 228, 228, 228, 228, 228, 228, 228, 228, 228, 228, 228, 228, 228, 228, 228, 228, 228, 228, 228, 228, 32,  
	32,  32,  32,  38,  38,  32,  228, 228, 228, 228, 228, 228, 228, 228, 228, 228, 228, 228, 228, 228, 228, 228, 228, 228, 228, 228, 32,  255, 255, 255, 32,  255, 255, 255, 32,  38,  228, 228, 228, 228, 228, 228, 228, 228, 228, 183, 183, 152, 152, 152, 152, 20,  20,  152, 228, 228, 228, 228, 228, 228, 228, 228, 228, 228, 228, 228, 228, 228, 228, 228, 228, 228, 228, 228, 152, 183, 183, 183, 152, 183, 183, 183, 152, 20,  228, 228, 228, 228, 228, 228, 228, 228, 228, 255, 255, 173, 173, 173, 173, 38,  38,  173, 228, 228, 228, 228, 228, 228, 228, 228, 228, 228, 228, 228, 228, 228, 228, 228, 228, 228, 228, 228, 173, 255, 255, 255, 173, 255, 255, 255, 173, 38,  228, 228, 228, 228, 228, 228, 228, 228, 228, 228, 228, 228, 228, 228, 228, 228, 228, 228, 228, 228, 228, 228, 228, 228, 228, 255, 255, 32,  
	32,  32,  32,  32,  32,  32,  228, 228, 228, 228, 228, 228, 228, 228, 228, 228, 228, 228, 228, 228, 228, 228, 228, 228, 228, 228, 32,  255, 255, 255, 255, 255, 255, 255, 32,  38,  32,  228, 228, 228, 228, 20,  228, 228, 152, 183, 183, 152, 152, 152, 152, 152, 152, 152, 228, 228, 228, 228, 228, 228, 228, 228, 228, 228, 228, 228, 228, 228, 228, 228, 228, 228, 228, 228, 152, 183, 183, 183, 183, 183, 183, 183, 152, 20,  152, 228, 228, 228, 228, 38,  228, 228, 173, 255, 255, 173, 173, 173, 173, 173, 173, 173, 228, 228, 228, 228, 228, 228, 228, 228, 228, 228, 228, 228, 228, 228, 228, 228, 228, 228, 228, 228, 173, 255, 255, 255, 255, 255, 255, 255, 173, 38,  173, 228, 228, 228, 228, 228, 228, 228, 228, 228, 228, 228, 228, 228, 228, 228, 228, 228, 228, 228, 228, 38,  228, 228, 32,  255, 255, 32,  
	32,  32,  32,  32,  32,  32,  32,  228, 228, 228, 228, 228, 228, 228, 228, 228, 228, 228, 228, 228, 228, 228, 228, 228, 228, 228, 32,  255, 255, 32,  255, 32,  255, 255, 32,  38,  32,  32,  228, 228, 20,  228, 20,  152, 152, 183, 183, 152, 152, 152, 152, 152, 152, 152, 152, 228, 228, 228, 228, 228, 228, 228, 228, 228, 228, 228, 228, 228, 228, 228, 228, 228, 228, 228, 152, 183, 183, 152, 183, 152, 183, 183, 152, 20,  152, 152, 228, 228, 38,  228, 38,  173, 173, 255, 255, 173, 173, 173, 173, 173, 173, 173, 173, 228, 228, 228, 228, 228, 228, 228, 228, 228, 228, 228, 228, 228, 228, 228, 228, 228, 228, 228, 173, 255, 255, 173, 255, 173, 255, 255, 173, 38,  173, 173, 228, 228, 228, 228, 228, 228, 228, 228, 228, 228, 228, 228, 228, 228, 228, 228, 228, 228, 38,  228, 38,  32,  32,  255, 255, 32,  
	32,  38,  32,  32,  32,  32,  32,  228, 228, 228, 228, 228, 228, 228, 228, 228, 228, 228, 228, 228, 228, 228, 228, 228, 228, 228, 32,  255, 255, 32,  255, 32,  255, 255, 32,  38,  32,  32,  228, 228, 20,  20,  20,  183, 183, 183, 152, 152, 152, 20,  152, 152, 152, 152, 152, 228, 228, 228, 228, 228, 228, 228, 228, 228, 228, 228, 228, 228, 228, 228, 228, 228, 228, 228, 152, 183, 183, 152, 183, 152, 183, 183, 152, 20,  152, 152, 228, 228, 38,  38,  38,  255, 255, 255, 173, 173, 173, 38,  173, 173, 173, 173, 173, 228, 228, 228, 228, 228, 228, 228, 228, 228, 228, 228, 228, 228, 228, 228, 228, 228, 228, 228, 173, 255, 255, 173, 255, 173, 255, 255, 173, 38,  173, 173, 228, 228, 228, 228, 228, 228, 228, 228, 228, 228, 228, 228, 228, 228, 228, 228, 228, 228, 38,  38,  38,  255, 255, 255, 32,  32,  
	38,  38,  38,  32,  32,  32,  32,  228, 228, 228, 228, 228, 228, 228, 228, 228, 228, 228, 228, 228, 228, 228, 228, 228, 228, 228, 228, 32,  32,  32,  38,  32,  32,  32,  32,  32,  32,  32,  228, 228, 20,  20,  20,  20,  183, 152, 152, 152, 20,  20,  20,  152, 152, 152, 152, 228, 228, 228, 228, 228, 228, 228, 228, 228, 228, 228, 228, 228, 228, 228, 228, 228, 228, 228, 228, 152, 152, 152, 20,  152, 152, 152, 152, 152, 152, 152, 228, 228, 38,  38,  38,  38,  255, 173, 173, 173, 38,  38,  38,  173, 173, 173, 173, 228, 228, 228, 228, 228, 228, 228, 228, 228, 228, 228, 228, 228, 228, 228, 228, 228, 228, 228, 228, 173, 173, 173, 38,  173, 173, 173, 173, 173, 173, 173, 228, 228, 228, 228, 228, 228, 228, 228, 228, 228, 228, 228, 228, 228, 228, 228, 228, 228, 38,  38,  38,  38,  255, 32,  32,  32,  
	255, 32,  38,  32,  32,  32,  32,  38,  255, 32,  32,  32,  255, 228, 228, 228, 228, 228, 228, 228, 228, 228, 228, 228, 228, 228, 228, 38,  38,  38,  32,  32,  32,  32,  38,  38,  38,  32,  32,  228, 183, 20,  20,  20,  152, 152, 20,  20,  183, 152, 20,  152, 152, 152, 152, 183, 183, 152, 152, 152, 183, 228, 228, 228, 228, 228, 228, 228, 228, 228, 228, 228, 228, 228, 228, 20,  20,  20,  152, 152, 152, 152, 20,  20,  20,  152, 152, 228, 255, 38,  38,  38,  173, 173, 38,  38,  255, 173, 38,  173, 173, 173, 173, 255, 255, 173, 173, 173, 255, 228, 228, 228, 228, 228, 228, 228, 228, 228, 228, 228, 228, 228, 228, 38,  38,  38,  173, 173, 173, 173, 38,  38,  38,  173, 173, 228, 228, 228, 228, 228, 228, 228, 228, 228, 228, 228, 228, 228, 228, 228, 228, 228, 255, 38,  38,  38,  32,  32,  38,  38,  
	32,  32,  38,  32,  32,  32,  32,  32,  255, 32,  32,  255, 255, 38,  32,  228, 228, 228, 228, 228, 228, 228, 228, 228, 228, 228, 38,  38,  38,  38,  38,  32,  32,  38,  38,  38,  38,  38,  32,  228, 228, 183, 228, 20,  20,  20,  20,  152, 152, 152, 20,  152, 152, 152, 152, 183, 183, 152, 152, 183, 183, 20,  152, 228, 228, 228, 228, 228, 228, 228, 228, 228, 228, 228, 20,  20,  20,  20,  20,  152, 152, 20,  20,  20,  20,  20,  152, 228, 228, 255, 228, 38,  38,  38,  38,  173, 173, 173, 38,  173, 173, 173, 173, 255, 255, 173, 173, 255, 255, 38,  173, 228, 228, 228, 228, 228, 228, 228, 228, 228, 228, 228, 38,  38,  38,  38,  38,  173, 173, 38,  38,  38,  38,  38,  173, 228, 228, 228, 13,  228, 228, 228, 228, 228, 228, 228, 228, 228, 228, 13,  228, 228, 228, 255, 228, 38,  38,  38,  38,  32,  
	32,  255, 38,  32,  32,  32,  32,  32,  255, 32,  255, 255, 255, 38,  38,  32,  255, 255, 255, 228, 228, 228, 228, 228, 228, 32,  38,  38,  38,  38,  38,  255, 255, 38,  38,  38,  38,  38,  32,  228, 228, 183, 228, 183, 183, 152, 183, 152, 152, 183, 20,  152, 152, 152, 152, 183, 183, 152, 183, 183, 183, 20,  20,  152, 183, 183, 183, 228, 228, 228, 228, 228, 228, 152, 20,  20,  20,  20,  20,  183, 183, 20,  20,  20,  20,  20,  152, 228, 228, 255, 228, 255, 255, 173, 255, 173, 173, 255, 38,  173, 173, 173, 173, 255, 255, 173, 255, 255, 255, 38,  38,  173, 255, 255, 255, 228, 228, 228, 228, 228, 228, 173, 38,  38,  38,  38,  38,  255, 255, 38,  38,  38,  38,  38,  173, 228, 228, 13,  38,  255, 255, 228, 228, 228, 228, 228, 228, 255, 255, 38,  13,  228, 228, 255, 228, 255, 255, 32,  255, 32,  
	32,  38,  38,  32,  32,  32,  228, 228, 255, 32,  32,  255, 38,  38,  32,  32,  255, 255, 38,  228, 228, 228, 228, 228, 228, 32,  255, 38,  38,  38,  255, 255, 255, 255, 38,  38,  38,  255, 32,  228, 228, 228, 228, 228, 183, 228, 228, 228, 152, 20,  20,  152, 152, 152, 183, 183, 183, 152, 152, 183, 20,  20,  152, 152, 183, 183, 20,  228, 228, 228, 228, 228, 228, 152, 183, 20,  20,  20,  183, 183, 183, 183, 20,  20,  20,  183, 152, 228, 228, 228, 228, 228, 255, 228, 228, 228, 173, 38,  38,  173, 173, 173, 255, 255, 255, 173, 173, 255, 38,  38,  173, 173, 255, 255, 38,  228, 228, 228, 228, 228, 228, 173, 255, 38,  38,  38,  255, 255, 255, 255, 38,  38,  38,  255, 173, 228, 228, 13,  13,  255, 228, 228, 228, 228, 228, 228, 228, 228, 255, 13,  13,  228, 228, 228, 228, 228, 255, 228, 228, 228, 
	228, 38,  38,  32,  32,  228, 228, 228, 32,  32,  32,  32,  32,  32,  32,  32,  32,  38,  38,  32,  255, 228, 228, 228, 228, 32,  255, 255, 255, 255, 255, 255, 255, 255, 255, 255, 255, 255, 32,  228, 228, 228, 228, 228, 228, 228, 228, 228, 183, 20,  20,  152, 152, 183, 183, 183, 152, 152, 152, 152, 152, 152, 152, 152, 152, 20,  20,  152, 183, 228, 228, 228, 228, 152, 183, 183, 183, 183, 183, 183, 183, 183, 183, 183, 183, 183, 152, 228, 228, 228, 228, 228, 228, 228, 228, 228, 255, 38,  38,  173, 173, 255, 255, 255, 173, 173, 173, 173, 173, 173, 173, 173, 173, 38,  38,  173, 255, 228, 228, 228, 228, 173, 255, 255, 255, 255, 255, 255, 255, 255, 255, 255, 255, 255, 173, 228, 228, 38,  13,  13,  255, 255, 228, 228, 228, 228, 255, 255, 13,  13,  38,  228, 228, 228, 228, 228, 228, 228, 228, 228, 
	38,  38,  32,  32,  228, 228, 228, 32,  32,  32,  32,  32,  32,  32,  32,  32,  32,  32,  32,  32,  255, 228, 228, 228, 32,  255, 255, 255, 255, 255, 255, 255, 255, 255, 255, 255, 255, 255, 255, 32,  228, 228, 228, 228, 228, 228, 228, 228, 20,  20,  152, 152, 183, 183, 183, 152, 152, 152, 152, 152, 152, 152, 152, 152, 152, 152, 152, 152, 183, 228, 228, 228, 152, 183, 183, 183, 183, 183, 183, 183, 183, 183, 183, 183, 183, 183, 183, 152, 228, 228, 228, 228, 228, 228, 228, 228, 38,  38,  173, 173, 255, 255, 255, 173, 173, 173, 173, 173, 173, 173, 173, 173, 173, 173, 173, 173, 255, 228, 228, 228, 173, 255, 255, 255, 255, 255, 255, 255, 255, 255, 255, 255, 255, 255, 255, 173, 13,  13,  13,  38,  255, 228, 228, 228, 228, 228, 228, 255, 38,  13,  13,  13,  228, 228, 228, 228, 228, 228, 228, 228, 
	38,  38,  32,  32,  228, 228, 255, 32,  32,  32,  32,  32,  32,  255, 255, 255, 32,  32,  32,  32,  32,  228, 228, 228, 32,  255, 255, 255, 255, 255, 32,  255, 255, 32,  255, 255, 255, 255, 255, 32,  228, 228, 228, 228, 228, 228, 183, 20,  20,  20,  152, 152, 183, 183, 183, 152, 152, 152, 152, 152, 152, 183, 183, 183, 152, 152, 152, 152, 152, 228, 228, 228, 152, 183, 183, 183, 183, 183, 152, 183, 183, 152, 183, 183, 183, 183, 183, 152, 228, 228, 228, 228, 228, 228, 255, 38,  38,  38,  173, 173, 255, 255, 255, 173, 173, 173, 173, 173, 173, 255, 255, 255, 173, 173, 173, 173, 173, 228, 228, 228, 173, 255, 255, 255, 255, 255, 173, 255, 255, 173, 255, 255, 255, 255, 255, 173, 38,  13,  13,  13,  13,  255, 255, 228, 228, 255, 255, 13,  13,  13,  13,  38,  228, 228, 228, 228, 228, 228, 255, 38,  
	38,  228, 228, 32,  32,  228, 228, 38,  255, 255, 255, 32,  32,  255, 255, 38,  32,  32,  32,  255, 255, 255, 228, 228, 32,  255, 255, 255, 255, 255, 32,  255, 255, 32,  255, 255, 255, 255, 255, 32,  228, 228, 228, 228, 228, 228, 228, 20,  20,  228, 228, 152, 152, 183, 183, 183, 183, 183, 183, 152, 152, 183, 183, 20,  152, 152, 152, 183, 183, 183, 228, 228, 152, 183, 183, 183, 183, 183, 152, 183, 183, 152, 183, 183, 183, 183, 183, 152, 228, 228, 228, 228, 228, 228, 228, 38,  38,  228, 228, 173, 173, 255, 255, 255, 255, 255, 255, 173, 173, 255, 255, 38,  173, 173, 173, 255, 255, 255, 228, 228, 173, 255, 255, 255, 255, 255, 173, 255, 255, 173, 255, 255, 255, 255, 255, 173, 13,  13,  38,  13,  13,  255, 228, 228, 228, 228, 255, 13,  13,  38,  13,  13,  228, 228, 228, 228, 228, 228, 228, 38,  
	228, 228, 228, 228, 228, 32,  32,  38,  38,  38,  255, 255, 32,  32,  38,  38,  32,  32,  32,  255, 255, 38,  228, 228, 32,  255, 255, 255, 255, 255, 32,  255, 255, 32,  255, 255, 255, 255, 255, 32,  228, 228, 228, 228, 228, 228, 228, 228, 228, 228, 228, 228, 228, 152, 152, 20,  20,  20,  183, 183, 152, 152, 20,  20,  152, 152, 152, 183, 183, 20,  228, 228, 152, 183, 183, 183, 183, 183, 152, 183, 183, 152, 183, 183, 183, 183, 183, 152, 228, 228, 228, 228, 228, 228, 228, 228, 228, 228, 228, 228, 228, 173, 173, 38,  38,  38,  255, 255, 173, 173, 38,  38,  173, 173, 173, 255, 255, 38,  228, 228, 173, 255, 255, 255, 255, 255, 173, 255, 255, 173, 255, 255, 255, 255, 255, 173, 13,  13,  13,  13,  13,  228, 228, 228, 228, 228, 228, 13,  13,  13,  13,  13,  228, 228, 228, 228, 228, 228, 228, 228, 
	228, 38,  38,  38,  228, 228, 228, 228, 38,  38,  38,  255, 255, 32,  32,  32,  32,  32,  32,  32,  38,  38,  228, 228, 32,  255, 255, 255, 255, 255, 255, 255, 255, 255, 255, 255, 255, 255, 255, 32,  228, 228, 228, 228, 228, 228, 228, 228, 228, 20,  20,  20,  228, 228, 183, 228, 20,  20,  20,  183, 183, 152, 152, 152, 152, 152, 152, 152, 20,  20,  228, 228, 152, 183, 183, 183, 183, 183, 183, 183, 183, 183, 183, 183, 183, 183, 183, 152, 228, 228, 228, 228, 228, 228, 228, 228, 228, 38,  38,  38,  228, 228, 255, 228, 38,  38,  38,  255, 255, 173, 173, 173, 173, 173, 173, 173, 38,  38,  228, 228, 173, 255, 255, 255, 255, 255, 255, 255, 255, 255, 255, 255, 255, 255, 255, 173, 13,  38,  13,  13,  13,  13,  255, 228, 228, 255, 13,  13,  13,  13,  38,  13,  228, 228, 228, 228, 228, 228, 228, 228, 
	38,  38,  228, 228, 38,  38,  38,  228, 255, 38,  38,  32,  255, 32,  32,  32,  32,  32,  32,  32,  32,  32,  255, 228, 32,  255, 32,  255, 255, 255, 255, 255, 255, 255, 255, 255, 255, 32,  255, 32,  228, 228, 228, 228, 228, 228, 228, 228, 20,  20,  183, 228, 20,  20,  20,  228, 183, 20,  20,  152, 183, 152, 152, 152, 152, 152, 152, 152, 152, 152, 183, 228, 152, 183, 152, 183, 183, 183, 183, 183, 183, 183, 183, 183, 183, 152, 183, 152, 228, 228, 228, 228, 228, 228, 228, 228, 38,  38,  255, 228, 38,  38,  38,  228, 255, 38,  38,  173, 255, 173, 173, 173, 173, 173, 173, 173, 173, 173, 255, 228, 173, 255, 173, 255, 255, 255, 255, 255, 255, 255, 255, 255, 255, 173, 255, 173, 13,  13,  13,  13,  38,  13,  255, 228, 228, 255, 13,  38,  13,  13,  13,  13,  228, 228, 228, 228, 228, 228, 228, 228, 
	38,  38,  228, 228, 38,  38,  38,  38,  228, 228, 32,  32,  255, 32,  32,  32,  255, 255, 255, 32,  32,  32,  38,  255, 228, 32,  255, 255, 255, 32,  255, 255, 255, 255, 32,  255, 255, 255, 32,  228, 228, 228, 228, 228, 228, 228, 228, 228, 20,  20,  228, 228, 20,  20,  20,  20,  228, 228, 152, 152, 183, 152, 152, 152, 183, 183, 183, 152, 152, 152, 20,  183, 228, 152, 183, 183, 183, 152, 183, 183, 183, 183, 152, 183, 183, 183, 152, 228, 228, 228, 228, 228, 228, 228, 228, 228, 38,  38,  228, 228, 38,  38,  38,  38,  228, 228, 173, 173, 255, 173, 173, 173, 255, 255, 255, 173, 173, 173, 38,  255, 228, 173, 255, 255, 255, 173, 255, 255, 255, 255, 173, 255, 255, 255, 173, 228, 228, 38,  13,  13,  13,  13,  13,  228, 228, 13,  13,  13,  13,  13,  38,  228, 228, 228, 228, 228, 228, 228, 228, 228, 
	38,  228, 228, 38,  38,  38,  38,  38,  38,  255, 32,  32,  255, 255, 32,  32,  255, 255, 38,  32,  32,  32,  228, 228, 228, 32,  255, 255, 255, 255, 32,  32,  32,  32,  255, 255, 255, 255, 32,  228, 228, 228, 228, 228, 228, 228, 228, 228, 20,  183, 228, 20,  20,  20,  20,  20,  20,  183, 152, 152, 183, 183, 152, 152, 183, 183, 20,  152, 152, 152, 228, 228, 228, 152, 183, 183, 183, 183, 152, 152, 152, 152, 183, 183, 183, 183, 152, 228, 228, 228, 228, 228, 228, 228, 228, 228, 38,  255, 228, 38,  38,  38,  38,  38,  38,  255, 173, 173, 255, 255, 173, 173, 255, 255, 38,  173, 173, 173, 228, 228, 228, 173, 255, 255, 255, 255, 173, 173, 173, 173, 255, 255, 255, 255, 173, 228, 228, 13,  13,  38,  13,  13,  13,  228, 228, 13,  13,  13,  38,  13,  13,  228, 228, 228, 228, 228, 228, 228, 228, 228, 
	38,  228, 228, 38,  38,  38,  38,  38,  32,  32,  32,  32,  255, 255, 32,  32,  32,  38,  38,  32,  32,  255, 255, 228, 228, 32,  255, 255, 255, 255, 255, 255, 255, 255, 255, 255, 255, 255, 32,  228, 228, 228, 228, 228, 228, 228, 228, 228, 20,  228, 228, 20,  20,  20,  20,  20,  152, 152, 152, 152, 183, 183, 152, 152, 152, 20,  20,  152, 152, 183, 183, 228, 228, 152, 183, 183, 183, 183, 183, 183, 183, 183, 183, 183, 183, 183, 152, 228, 228, 228, 228, 228, 228, 228, 228, 228, 38,  228, 228, 38,  38,  38,  38,  38,  173, 173, 173, 173, 255, 255, 173, 173, 173, 38,  38,  173, 173, 255, 255, 228, 228, 173, 255, 255, 255, 255, 255, 255, 255, 255, 255, 255, 255, 255, 173, 228, 228, 228, 13,  13,  13,  13,  38,  228, 228, 38,  13,  13,  13,  13,  228, 228, 228, 228, 228, 228, 228, 228, 228, 228, 
	228, 228, 228, 38,  38,  38,  38,  38,  32,  32,  32,  32,  255, 255, 32,  32,  32,  32,  32,  32,  32,  255, 255, 255, 228, 228, 32,  255, 255, 255, 255, 32,  32,  255, 255, 255, 255, 32,  228, 228, 228, 228, 228, 228, 228, 228, 228, 228, 228, 183, 228, 20,  20,  20,  20,  20,  152, 152, 152, 152, 183, 183, 152, 152, 152, 152, 152, 152, 152, 183, 183, 183, 228, 228, 152, 183, 183, 183, 183, 152, 152, 183, 183, 183, 183, 152, 228, 228, 228, 228, 228, 228, 228, 228, 228, 228, 228, 255, 228, 38,  38,  38,  38,  38,  173, 173, 173, 173, 255, 255, 173, 173, 173, 173, 173, 173, 173, 255, 255, 255, 228, 228, 173, 255, 255, 255, 255, 173, 173, 255, 255, 255, 255, 173, 228, 228, 228, 228, 228, 13,  38,  13,  13,  13,  13,  13,  13,  38,  13,  228, 228, 228, 228, 228, 228, 228, 228, 228, 228, 228, 
	228, 228, 228, 228, 38,  38,  38,  228, 32,  32,  32,  32,  32,  255, 32,  32,  32,  32,  255, 255, 32,  38,  38,  228, 228, 228, 228, 32,  32,  32,  32,  228, 228, 32,  32,  32,  32,  228, 228, 228, 228, 228, 228, 228, 228, 228, 228, 228, 228, 228, 228, 183, 20,  20,  20,  228, 152, 152, 152, 152, 152, 183, 152, 152, 152, 152, 183, 183, 152, 20,  20,  228, 228, 228, 228, 152, 152, 152, 152, 228, 228, 152, 152, 152, 152, 228, 228, 228, 228, 228, 228, 228, 228, 228, 228, 228, 228, 228, 228, 255, 38,  38,  38,  228, 173, 173, 173, 173, 173, 255, 173, 173, 173, 173, 255, 255, 173, 38,  38,  228, 228, 228, 228, 173, 173, 173, 173, 228, 228, 173, 173, 173, 173, 228, 228, 228, 228, 228, 228, 228, 228, 13,  38,  13,  13,  38,  13,  228, 228, 228, 228, 228, 228, 228, 228, 228, 228, 228, 228, 228, 
	228, 228, 228, 228, 228, 228, 228, 228, 32,  32,  32,  32,  32,  255, 255, 32,  32,  32,  38,  32,  32,  38,  32,  228, 228, 228, 228, 228, 228, 228, 228, 228, 228, 228, 228, 228, 228, 228, 228, 228, 228, 228, 228, 228, 228, 228, 228, 228, 228, 228, 228, 228, 228, 228, 228, 228, 152, 152, 152, 152, 152, 183, 183, 152, 152, 152, 20,  152, 152, 20,  152, 228, 228, 228, 228, 228, 228, 228, 228, 228, 228, 228, 228, 228, 228, 228, 228, 228, 228, 228, 228, 228, 228, 228, 228, 228, 228, 228, 228, 228, 228, 228, 228, 228, 173, 173, 173, 173, 173, 255, 255, 173, 173, 173, 38,  173, 173, 38,  173, 228, 228, 228, 228, 228, 228, 228, 228, 228, 228, 228, 228, 228, 228, 228, 228, 228, 38,  38,  228, 228, 228, 228, 228, 38,  38,  228, 228, 228, 228, 228, 38,  38,  228, 228, 228, 228, 228, 228, 228, 228, 
	228, 228, 228, 228, 228, 228, 228, 228, 228, 32,  32,  32,  32,  32,  255, 255, 255, 32,  32,  32,  32,  32,  32,  228, 228, 228, 228, 228, 228, 228, 228, 228, 228, 228, 228, 228, 228, 228, 228, 228, 228, 228, 228, 228, 228, 228, 228, 228, 228, 228, 228, 228, 228, 228, 228, 228, 228, 152, 152, 152, 152, 152, 183, 183, 183, 152, 152, 152, 152, 152, 152, 228, 228, 228, 228, 228, 228, 228, 228, 228, 228, 228, 228, 228, 228, 228, 228, 228, 228, 228, 228, 228, 228, 228, 228, 228, 228, 228, 228, 228, 228, 228, 228, 228, 228, 173, 173, 173, 173, 173, 255, 255, 255, 173, 173, 173, 173, 173, 173, 228, 228, 228, 228, 228, 228, 228, 228, 228, 228, 228, 228, 228, 228, 228, 228, 228, 38,  13,  38,  38,  228, 228, 228, 38,  38,  228, 228, 228, 38,  38,  13,  38,  228, 228, 228, 228, 228, 228, 228, 228, 
	228, 228, 228, 228, 228, 228, 228, 228, 228, 228, 32,  32,  32,  32,  32,  255, 255, 255, 255, 32,  32,  32,  32,  228, 228, 228, 228, 228, 228, 228, 228, 228, 228, 228, 228, 228, 228, 228, 228, 228, 228, 228, 228, 228, 228, 228, 228, 228, 228, 228, 228, 228, 228, 228, 228, 228, 228, 228, 152, 152, 152, 152, 152, 183, 183, 183, 183, 152, 152, 152, 152, 228, 228, 228, 228, 228, 228, 228, 228, 228, 228, 228, 228, 228, 228, 228, 228, 228, 228, 228, 228, 228, 228, 228, 228, 228, 228, 228, 228, 228, 228, 228, 228, 228, 228, 228, 173, 173, 173, 173, 173, 255, 255, 255, 255, 173, 173, 173, 173, 228, 228, 228, 228, 228, 228, 228, 228, 228, 228, 228, 228, 228, 228, 228, 228, 228, 228, 38,  13,  38,  38,  228, 228, 38,  38,  228, 228, 38,  38,  13,  38,  228, 228, 228, 228, 228, 228, 228, 228, 228, 
	228, 228, 228, 228, 228, 228, 228, 228, 228, 228, 38,  32,  32,  32,  32,  32,  32,  255, 255, 255, 255, 255, 255, 255, 228, 228, 228, 228, 228, 228, 228, 228, 228, 228, 228, 228, 228, 228, 228, 228, 228, 228, 228, 228, 228, 228, 228, 228, 228, 228, 228, 228, 228, 228, 228, 228, 228, 228, 20,  152, 152, 152, 152, 152, 152, 183, 183, 183, 183, 183, 183, 183, 228, 228, 228, 228, 228, 228, 228, 228, 228, 228, 228, 228, 228, 228, 228, 228, 228, 228, 228, 228, 228, 228, 228, 228, 228, 228, 228, 228, 228, 228, 228, 228, 228, 228, 38,  173, 173, 173, 173, 173, 173, 255, 255, 255, 255, 255, 255, 255, 228, 228, 228, 228, 228, 228, 228, 228, 228, 228, 228, 228, 228, 228, 228, 228, 228, 38,  38,  13,  38,  38,  228, 38,  38,  228, 38,  38,  13,  38,  38,  228, 228, 228, 228, 228, 228, 228, 228, 228, 
	228, 228, 228, 228, 228, 228, 228, 228, 228, 255, 255, 38,  38,  32,  32,  38,  38,  38,  38,  255, 255, 255, 255, 255, 228, 228, 228, 228, 228, 228, 228, 228, 228, 228, 228, 228, 228, 228, 228, 228, 228, 228, 228, 228, 228, 228, 228, 228, 228, 228, 228, 228, 228, 228, 228, 228, 228, 183, 183, 20,  20,  152, 152, 20,  20,  20,  20,  183, 183, 183, 183, 183, 228, 228, 228, 228, 228, 228, 228, 228, 228, 228, 228, 228, 228, 228, 228, 228, 228, 228, 228, 228, 228, 228, 228, 228, 228, 228, 228, 228, 228, 228, 228, 228, 228, 255, 255, 38,  38,  173, 173, 38,  38,  38,  38,  255, 255, 255, 255, 255, 228, 228, 228, 228, 228, 228, 228, 228, 228, 228, 228, 228, 228, 228, 228, 228, 228, 228, 38,  38,  13,  38,  228, 38,  38,  228, 38,  13,  38,  38,  228, 228, 228, 228, 228, 228, 228, 228, 228, 228, 
	228, 228, 228, 228, 228, 228, 228, 228, 255, 255, 255, 38,  38,  38,  38,  38,  38,  38,  38,  38,  38,  255, 255, 228, 228, 228, 228, 228, 228, 228, 228, 228, 228, 228, 228, 228, 228, 228, 228, 228, 228, 228, 228, 228, 228, 228, 228, 228, 228, 228, 228, 228, 228, 228, 228, 228, 183, 183, 183, 20,  20,  20,  20,  20,  20,  20,  20,  20,  20,  183, 183, 228, 228, 228, 228, 228, 228, 228, 228, 228, 228, 228, 228, 228, 228, 228, 228, 228, 228, 228, 228, 228, 228, 228, 228, 228, 228, 228, 228, 228, 228, 228, 228, 228, 255, 255, 255, 38,  38,  38,  38,  38,  38,  38,  38,  38,  38,  255, 255, 228, 228, 228, 228, 228, 228, 228, 228, 228, 228, 228, 228, 228, 228, 228, 228, 228, 228, 228, 38,  38,  38,  13,  38,  38,  38,  38,  13,  38,  38,  38,  228, 228, 228, 228, 228, 228, 228, 228, 228, 228, 
	228, 228, 228, 228, 228, 228, 228, 228, 228, 228, 228, 228, 255, 255, 38,  38,  255, 255, 38,  38,  38,  38,  228, 228, 228, 228, 228, 228, 228, 228, 228, 228, 228, 228, 228, 228, 228, 228, 228, 228, 228, 228, 228, 228, 228, 228, 228, 228, 228, 228, 228, 228, 228, 228, 228, 228, 228, 228, 228, 228, 183, 183, 20,  20,  183, 183, 20,  20,  20,  20,  228, 228, 228, 228, 228, 228, 228, 228, 228, 228, 228, 228, 228, 228, 228, 228, 228, 228, 228, 228, 228, 228, 228, 228, 228, 228, 228, 228, 228, 228, 228, 228, 228, 228, 228, 228, 228, 228, 255, 255, 38,  38,  255, 255, 38,  38,  38,  38,  228, 228, 228, 228, 228, 228, 228, 228, 228, 228, 228, 228, 228, 228, 228, 228, 228, 228, 228, 228, 228, 38,  38,  38,  38,  38,  38,  38,  38,  38,  38,  228, 228, 228, 228, 228, 228, 228, 228, 228, 228, 228, 
	228, 228, 228, 228, 228, 228, 228, 228, 228, 228, 228, 255, 255, 255, 38,  255, 255, 255, 38,  38,  38,  38,  38,  228, 228, 228, 228, 228, 228, 228, 228, 228, 228, 228, 228, 228, 228, 228, 228, 228, 228, 228, 228, 228, 228, 228, 228, 228, 228, 228, 228, 228, 228, 228, 228, 228, 228, 228, 228, 183, 183, 183, 20,  183, 183, 183, 20,  20,  20,  20,  20,  228, 228, 228, 228, 228, 228, 228, 228, 228, 228, 228, 228, 228, 228, 228, 228, 228, 228, 228, 228, 228, 228, 228, 228, 228, 228, 228, 228, 228, 228, 228, 228, 228, 228, 228, 228, 255, 255, 255, 38,  255, 255, 255, 38,  38,  38,  38,  38,  228, 228, 228, 228, 228, 228, 228, 228, 228, 228, 228, 228, 228, 228, 228, 228, 228, 228, 228, 228, 228, 228, 228, 38,  38,  38,  38,  228, 228, 228, 228, 228, 228, 228, 228, 228, 228, 228, 228, 228, 228 
}	% Conjunto de bytes de cor da sprite no tilemap

% Pixel em perspectiva
% arg1 = byte de cor
% arg2 = indice de posição
% arg3 = poto superior esqeerdo do tilemap
% arg4 = sprite length
% arg5 = pixel opacity
\newcommand{\getr}[1]{\DIVIDE{#1}{7}{\r}}
%
\newcommand{\getgbits}[1]{\INTEGERDIVISION{#1}{8}{\dummy}{\gbits}}
\newcommand{\getg}[1]{\DIVIDE{#1}{7}{\g}}
%
\newcommand{\getbbits}[1]{\INTEGERDIVISION{#1}{64}{\bbits}{\dummy}}
\newcommand{\getb}[1]{\DIVIDE{#1}{3}{\b}}
%
\newcommand{\getxy}[2]{\INTEGERDIVISION{#1}{#2}{\y}{\x}}

\newcommand{\slantedPixel}[5]{
	%
	\INTEGERDIVISION{#1}{64}{\b}{\dummy}
	\INTEGERDIVISION{#1}{8}{\bg}{\rbits}
		
	\getbbits{#1}	\getb{\bbits}
	\getgbits{\bg}	\getg{\gbits}
					\getr{\rbits}
	
	\definecolor{byte color}{rgb}{\r, \g, \b}
	 
	\getxy{#2}{#4}	% 
	
	\fill [color=byte color, opacity=#5] 
		#3 ++ (\angle:\x) ++ (0, -\y) --++ 
		(\angle:1) --++ (90:-1) --++ (\angle:-1) --++ (90:1)
	;
}

% Tela mostrada ao jogador
\newcommand{\window}[2]{
	\begin{scope}[shift = {(#2)}]
		\draw [#1] 
			(0:0) 				node (down left) {}--++ 
			(\angle:\bmpLen)	node (down right) {} --++ 
			(90:\tilemapHeight) node (up right) {} --++ 
			(\angle:-\bmpLen)	node (up left) {}
			-- cycle
		;
		\foreach \x in {1, 2, ..., 19} {
			\draw [thin, opacity=.25] (\angle:\x*\tileLen) --++ (90:\tilemapHeight);
		}
		\foreach \y in {1, 2, ..., 14} {
			\draw [thin, opacity=.25] (90:\y*\tileLen) --++ (\angle:\bmpLen);
		}
	\end{scope}
}

% Tilemap inteiro
\newcommand{\tilemap}[2]{
	\begin{scope}[shift = {(#2)}]
		\draw [#1]
			(0:0) --++ 
			(\angle:\tilemapLen) --++ 
			(90:\tilemapHeight) --++ 
			(\angle:-\tilemapLen) node (tilemapStart) {} -- 
			cycle
		;
		\foreach \x in {1, 2, ..., 29} {
			\draw [thin, opacity=.1] (\angle:\x*\tileLen) --++ (90:\tilemapHeight);
		}
		\foreach \y in {1, 2, ..., 14} {
			\draw [thin, opacity=.1] (90:\y*\tileLen) --++ (\angle:\tilemapLen);
		}
	\end{scope}
}
